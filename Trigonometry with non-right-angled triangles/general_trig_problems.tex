\newpage
\subsection{Problems involving non-right-angled triangles}

\begin{examplecz}{}
Three sides of a triangle measure $7$\,cm, $9$\,cm and $12$\,cm respectivey.
\vskip3mm
Find the area of the triangle, correct to one decimal place.
\tcblower
\textbf{Solution:}
\vspace*{6.5cm}
\end{examplecz}

\vspace{5mm}

\begin{examplecz}{}
In triangle $AMN$ shown, $AM=15$\,cm, $MN=12.3$\,cm and $\angle NAM=42^{\circ}$ as shown.
\begin{center}
    \begin{tikzpicture}
        \coordinate[label=below left: $A$] (A) at (0,0);
        \coordinate[label=below right: $M$] (M) at (6,0);
        \coordinate[label=above right: $N$] (N) at (5,3.5);
        
        \draw (A) -- node[below]{$15$\,cm} (M) -- node[above right]{$12.3$\,cm}  (N) --  cycle;
        \tkzLabelAngle[pos=1.2](M,A,N){$42^{\circ}$};
        %\tkzLabelAngle[pos=1.1](H,I,J){$52^{\circ}$};        
    \end{tikzpicture}
\end{center}
Find the area of the triangle, correct to two decimal places.
\tcblower
\textbf{Solution:}
\vspace*{4.5cm}
%\begin{align*}
%    \dfrac{\sin \theta}{13} &= \dfrac{\sin 89 ^{\circ}}{25} \\
%    \sin \theta &= \dfrac{13 \sin 89^{\circ}}{25} \\
%    \therefore \theta &= \sin ^{-1} \left(\dfrac{13 \sin 89^{\circ}}{25}  \right) \\
%    &= 31^{\circ}
%\end{align*}
\end{examplecz}

\vspace{5mm}

\begin{examplecz}{}
In an acute angled triangle $ABC$, $b=14$\,cm, $c=16$\,cm and $\cos A= \dfrac{7}{10}$.
\vskip3mm
Find the exact area of the triangle.
\tcblower
\textbf{Solution:}
\vspace*{7.5cm}
%\begin{align*}
%    \dfrac{\sin \theta}{13} &= \dfrac{\sin 89 ^{\circ}}{25} \\
%    \sin \theta &= \dfrac{13 \sin 89^{\circ}}{25} \\
%    \therefore \theta &= \sin ^{-1} \left(\dfrac{13 \sin 89^{\circ}}{25}  \right) \\
%    &= 31^{\circ}
%\end{align*}
\end{examplecz}

\newpage

\begin{examplecz}[height fill=true]{}
Two surveying drones depart from point $P$ at the same time. Drone $A$ flies on a bearing of $280^{\circ}$ at $60$\,m/s and drone $B $ flies on a bearing of $015^{\circ}$ at $50$\,m/s. After $2$ minutes, the drones stop and hold their position.
\vskip3mm
\begin{enumerate}
    \item [(a)] Find the distance between the drones when they stop.
    \item  [(b)] Find the bearing from drone $A$ to drone $B$ after $2$ minutes.
    \item [(c)] Find the area enclosed by the edges joining point $P$ and the locations of the drones after $2$ minutes.  
\end{enumerate}
\tcblower
\textbf{Solution:}
\vspace{2cm}
%\begin{align*}
%    \dfrac{\sin \theta}{13} &= \dfrac{\sin 89 ^{\circ}}{25} \\
%    \sin \theta &= \dfrac{13 \sin 89^{\circ}}{25} \\
%    \therefore \theta &= \sin ^{-1} \left(\dfrac{13 \sin 89^{\circ}}{25}  \right) \\
%    &= 31^{\circ}
%\end{align*}
\end{examplecz}

\newpage

\begin{examplecz}[height fill=true]{}

Let points $P$ and $Q$ denote the top and bottom of a tower respectively. Point $A$ is due east of a tower. From another point $B$, the tower is on a bearing of $051^{\circ}$. The angles of elevation to $P$ from $A$ and $B$ are $12^{\circ}$ and $11^{\circ}$. Points $A$ and $B$ are $1000$ metres apart as shown.
\begin{center}
    \begin{tikzpicture}
        \coordinate[label = below: $Q$] (Q) at (0,0);
        \coordinate[label = above: $P$] (P) at (0,3);
        \coordinate[label = left: $B$] (B) at (-3,-2);
        \coordinate[label = right: $A$] (A) at (5,0);

        \draw (B) -- (P) -- (A) -- node[below right]{$1000$\,m} cycle;
        \draw[dashed] (P) -- node[right]{$h$} (Q);
        \draw[dashed] (B) -- (Q) -- (A);

        \tkzLabelAngle[pos=1.4](Q,B,P){$11^{\circ}$};
        \tkzLabelAngle[pos=1.4](P,A,Q){$12^{\circ}$};
        \tkzMarkRightAngle[size=0.35](P,Q,B);
        \tkzMarkRightAngle[size=0.35](A,Q,P);
        %\tkzMarkAngle[size=0.75cm,color=cyan,label=$\theta$](B,P,A);        
    \end{tikzpicture}
\end{center}
\begin{enumerate}
    \item Show that $\angle AQB = 141^{\circ}$.
    \item Using $\Delta APQ$, show that $AQ=h\tan 78^{\circ}$, where $h$ is the height of the tower.
    \item Hence, find the value of $h$.
\end{enumerate}

\tcblower
\textbf{Solution:}
%\vspace*{12cm}
%\begin{align*}
%    \dfrac{\sin \theta}{13} &= \dfrac{\sin 89 ^{\circ}}{25} \\
%    \sin \theta &= \dfrac{13 \sin 89^{\circ}}{25} \\
%    \therefore \theta &= \sin ^{-1} \left(\dfrac{13 \sin 89^{\circ}}{25}  \right) \\
%    &= 31^{\circ}
%\end{align*}
\end{examplecz}