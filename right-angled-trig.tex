\section{Primitive Functions}

\subsection{The Anti-Derivative}

\begin{theorybox}{Notes}
    The sinusoidal functions refer to the sine and cosine functions, $y= \sin x$ and $y=\cos x$ whose graphs are shown below.

    \begin{center}
    \begin{tikzpicture}[>=stealth]
        \begin{axis}[
            standard,
            xlabel = {$x$},
            ylabel = {$y$},
            xmin=-2, xmax=7.7,
            ymin=-1.25, ymax=1.25,
            xtick={1.570796327, 3.141592654, 4.71238898, 6.283185307}, 
            xticklabels={$\frac{\pi}{2}$, $\pi$, $\frac{3\pi}{2}$, $2\pi$}, 
            ytick={-1,1},
            yticklabels={$-1$, $1$},
            height=7cm,
            width=14cm
            ]
        \coordinate[label=below left:$O$] (O) at (0,0);
    
        % Plot 1
        \addplot [samples=1200, domain = 0:6.2831853] { sin(x) };
        \addplot [ dashed, samples=400, domain = -1:0] { sin(x) };
        \addplot [ dashed, samples=400, domain = 6.2831853:7] { sin(x) } node[above]{$y=\sin x$}; 
        
        \draw [blue, dashed, latex-latex] (-1,0) --  node[left, blue]{amplitude} (-1,1) ;
        \draw [blue, dashed, latex-latex] (-1,-1) --  node[left, blue]{amplitude} (-1,0) ;

        \draw [purple, dashed, latex-latex] (0, 1.15) -- node[above, purple]{period} (6.283185307, 1.15);
        \end{axis}
    \end{tikzpicture}
    \end{center}

    \begin{center}
    \begin{tikzpicture}[>=stealth]
        \begin{axis}[
            standard,
            xlabel = {$x$},
            ylabel = {$y$},
            xmin=-2, xmax=7.7,
            ymin=-1.25, ymax=1.25,
            xtick={1.570796327, 3.141592654, 4.71238898, 6.283185307}, 
            xticklabels={$\frac{\pi}{2}$, $\pi$, $\frac{3\pi}{2}$, $2\pi$}, 
            ytick={-1,1},
            yticklabels={$-1$, $1$},
            height=7cm,
            width=14cm
            ]
        \coordinate[label=below left:$O$] (O) at (0,0);
    
        % Plot 1
        \addplot [samples=1200, domain = 0:6.2831853] { cos(x) };
        \addplot [ dashed, samples=400, domain = -1:0] { cos(x) };
        \addplot [ dashed, samples=400, domain = 6.2831853:7] { cos(x) } node[below]{$y=\cos x$}; 

        \draw [blue, dashed, latex-latex] (-1,0) --  node[left, blue]{amplitude} (-1,1) ;
        \draw [blue, dashed, latex-latex] (-1,-1) --  node[left, blue]{amplitude} (-1,0) ;

        \draw [purple, dashed, latex-latex] (0, 1.15) -- node[above, purple]{period} (6.283185307, 1.15);
        \end{axis}
    \end{tikzpicture}
    \end{center}
    Transformations to the functions above can be expressed by writing them in the form
    \begin{center}
        $ y = a \sin \left( b \left(x-c \right) \right) + d $ and $ y = a \cos \left( b \left(x-c \right) \right) + d $
    \end{center}
    where:
    \begin{itemize}
        \item $a$ is the (magnitude of the) amplitude;
        \item the period, $T$, is given by $T= \dfrac{2\pi}{b}$;
        \vskip3mm
        Hint: when drawing sinusoidal graphs, calculate the period, then mark it out using an interval that is divisible by $4$ on the horizontal axis.
        \item $c$ is the phase (horizontal) shift; and
        \item $d$ is the principal (central) axis (or equilibrium).
        
    \end{itemize}
\end{theorybox}

\begin{examplecz}{}
    On separate number planes, sketch the graphs of:
    \begin{enumerate}
    \begin{multicols}{2}
        \item $ y = 2 \sin (x)$
        \item $ y = -3 \cos (x) + 3$
        \item $ y = 3 \sin ( 2x ) $
        \item $ y = 1 - \cos ( 3x ) $
    \end{multicols}
    \end{enumerate}
\end{examplecz}

\vspace*{5mm}

\begin{examplecz}{}
    On separate number planes, sketch the graphs of:
    \begin{enumerate}
    \begin{multicols}{2}
        \item $ y =  \cos \br{2 \br{x- \dfrac{\pi}{4}} } + 1 $
        \item $ y =  - \dfrac{3}{2} \sin \br{ \dfrac{x}{3} + \dfrac{\pi}{2} }$
        \item $ y =  \cos \br{ \pi \br{x+3} } - 2 $
        \item $ y =  3-5\sin \br{ \dfrac{\pi}{2} \br{x+3}} $
    \end{multicols}
    \end{enumerate}
\end{examplecz}

\vspace*{5mm}

\begin{examplecz}{2022 HSC Mathematics Advanced Question 14}
    The graph of $y = k \sin (ax)$ is shown. \hfill \textbf{2}
    \begin{center}
    \begin{tikzpicture}[>=stealth]
        \begin{axis}[
            standard,
            xlabel = {$x$},
            ylabel = {$y$},
            xmin=-25.13274123, xmax=25.13274123,
            ymin=-4, ymax=4,
            xtick={-25.13274123, -18.84955592, -12.56637061, -6.283185307, 6.283185307, 12.56637061, 18.84955592, 25.13274123}, 
            xticklabels={$ - 8 \pi$, $-6\pi$, $-4\pi$, $-2\pi$, $2\pi$, $4\pi$, $6\pi$, $8\pi$}, 
            ytick={-4, -2, 2, 4},
            yticklabels={$-4$, $-2$, $2$, $4$},
            height=6cm,
            width=17.5cm
            ]
        \coordinate[label=below left:$O$] (O) at (0,0);
    
        % Plot 1
        \addplot [samples=2000, domain = -25.13274123:25.13274123] { 4*sin((1/3)*x) };
        \end{axis}
    \end{tikzpicture}
    \end{center}
    What are the values of $a$ and $k$?
\end{examplecz}

\vspace*{5mm}

\begin{examplecz}{}
    Let $ f \br{x} = 5.8 \sin \br{\dfrac{\pi}{6} \br{x+1}} +b $. The graph of $y=f\br{x}$ has a local maximum at $ \br{2,\,21.8}$ and a local minimum at $ \br{8,\,10.2}$.
    \begin{enumerate}
        \item Find the period of $f$.
        \item Find the value of $b$.
    \end{enumerate}
\end{examplecz}

\begin{examplecz}{}
    Let $ f \br{x} = p \sin \br{\dfrac{2\pi}{9} \br{x-3.75}} +q $. The graph of $y=f\br{x}$ passes through the points $\br{3,\,2.5}$ and $6,\,15.1$.
    \vskip3mm
    Find the values of $p$ and $q$.
\end{examplecz}

\clearpage

\subsection{Boundary Conditions}

\begin{theorybox}{Notes}
    The graph of $y= \tan x$ is shown below.

    \begin{center}
    \begin{tikzpicture}[>=stealth]
        \begin{axis}[
            standard,
            xlabel = {$x$},
            ylabel = {$y$},
            xmin=-3.1, xmax=6.3,
            ymin=-4, ymax=4,
            xtick={-3.141592654, -1.570796327, 0.7853981634, 1.570796327, 3.141592654, 4.71238898, 6.283185307}, 
            xticklabels={$-\pi$,$-\frac{\pi}{2}$,$\frac{\pi}{4}$, $\frac{\pi}{2}$, $\pi$, $\frac{3\pi}{2}$, $2\pi$}, 
            ytick={1},
            yticklabels={$1$},
            height=10cm,
            width=16cm
            ]
        \coordinate[label=below left:$O$] (O) at (0,0);
    
        % Plot 1
        \addplot [samples=800, domain = -3.141592654:-1.81577] { tan(x) };
        \addplot [samples=800, domain = -1.32582:1.32582] { tan(x) };
        \addplot [samples=800, domain = 1.81577:4.46741] { tan(x) };
        \addplot [samples=800, domain = 4.95737:6.283185307] { tan(x) };

        \draw [dashed] (-1.570796327, -4) -- (-1.570796327, 4);
        \draw [dashed] (1.570796327, -4) -- (1.570796327, 4);
        \draw [dashed] (4.71238898, -4) -- (4.71238898, 4);

        \fill (axis cs: 0.7853981634,1) circle (2pt);

        \draw [purple, dashed, latex-latex] (-1.570796327, -4.15) -- node[below, purple]{period} (1.570796327, -4.15);
        \end{axis}
    \end{tikzpicture}
    \end{center}

    Transformations to the functions above can be expressed by writing them in the form
    \begin{center}
        $ y = a \tan \left( b \left(x-c \right) \right) + d $
    \end{center}
    where:
    \begin{itemize}
        \item the period, $T$, is given by $T= \dfrac{\pi}{b}$; and
        \item $c$ is the phase (horizontal) shift.        
    \end{itemize}
\end{theorybox}

\begin{examplecz}{}
    On separate number planes, sketch the graphs of:
    \begin{enumerate}
    \begin{multicols}{3}
        \item $ y =  \tan \br{2x} $
        \item $ y =  - \tan \br{ \dfrac{x}{2} }$
        \item $ y =  \tan \br{ 2 \br{x - \dfrac{\pi}{4}} } $
    \end{multicols}
    \end{enumerate}
\end{examplecz}



%\vfill

%\begin{Qbox}{Homework}
%    From CambridgeMATHS - Mathematics Extension 1
%    \begin{itemize}
%    \begin{multicols}{2}
%        \item Exercise 6C: 1 to 3, 6, 8, 10, 11 and 13
%    \end{multicols}
%    \end{itemize}
    
%\end{Qbox}

\clearpage