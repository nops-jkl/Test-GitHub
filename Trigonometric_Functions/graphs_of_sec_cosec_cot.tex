\newpage
\subsection*{Graphs of the Reciprocal Ratios}

\begin{theorybox}{}
The diagrams below show the graphs of $y= \text{cosec}\,{x}$, $y=\sec x$ and $ y = \cot x$ for $ 0 \leq x \leq 2\pi$.

\begin{center}
    \begin{tikzpicture}[>=stealth]
        \begin{axis}[
            clip=false,
            standard,
            xlabel = {$x$},
            ylabel = {$y$},
            xmin=0, xmax=2*pi,
            ymin=-3, ymax=3,
            xtick={pi/2, 3*pi/2}, 
            xticklabels={$\frac{\pi}{2}$, $\frac{3\pi}{2}$}, 
            ytick={-1,1},
            yticklabels={$-1$, $1$},
            height=8cm,
            width=12cm
            ]
        \coordinate[label=below left:$O$] (O) at (0,0);
    
        % Plot 1
        \addplot [samples=750, domain = 0.34:2.80] { cosec(x) } node[ above right]{$y=\text{cosec}\,x$};
       \addplot [samples=750, domain = 3.48:5.94] { cosec(x) };
        \draw [dashed,] (pi,-3) -- (pi,0)  node[below left]{$\pi$} -- (pi,3) ;
        \draw [dashed,] (2*pi,-3) -- (2*pi,0)  node[below left]{$2\pi$} -- (2*pi,3) ;
        \end{axis}
    \end{tikzpicture}
    \end{center}

\begin{center}
    \begin{tikzpicture}[>=stealth]
        \begin{axis}[
            clip=false,
            standard,
            xlabel = {$x$},
            ylabel = {$y$},
            xmin=0, xmax=2*pi,
            ymin=-3, ymax=3,
            xtick={pi, 2*pi}, 
            xticklabels={$\pi$, $2\pi$}, 
            ytick={-1,1},
            yticklabels={$-1$, $1$},
            height=8cm,
            width=12cm
            ]
        \coordinate[label=below left:$O$] (O) at (0,0);
    
        % Plot 1
        \addplot [samples=500, domain = 0:1.23] { sec(x) } node[ above right]{$y=\text{sec}\,x$};
        \addplot [samples=500, domain = 1.91:4.37] { sec(x) };
        \addplot [samples=500, domain = 5.05:2*pi] { sec(x) };
        \draw [dashed,] (pi/2,-3) -- (pi/2,0)  node[below left]{$\frac{\pi}{2}$} -- (pi/2,3) ;
        \draw [dashed,] (3*pi/2,-3) -- (3*pi/2,0)  node[below left]{$\frac{3\pi}{2}$} -- (3*pi/2,3) ;
        \end{axis}
    \end{tikzpicture}
    \end{center}

    \begin{center}
    \begin{tikzpicture}[>=stealth]
        \begin{axis}[
            standard,
            xlabel = {$x$},
            ylabel = {$y$},
            xmin=0, xmax=2*pi,
            ymin=-4, ymax=4,
            xtick={pi/2, 3*pi/2}, 
            xticklabels={ $\frac{\pi}{2}$, $\frac{3\pi}{2}$}, 
            ytick={0},
            yticklabels={},
            height=8cm,
            width=12cm
            ]
        \coordinate[label=below left:$O$] (O) at (0,0);
    
        % Plot 1
        \addplot [samples=250, domain = 0.24:2.897] { cot(x) };
        \addplot [samples=250, domain = 3.386:6.038] { cot(x) } node[below left]{$y=\cot x$};


        \draw [dashed] (pi,-4) -- (pi,0) node[below left]{$\frac{\pi}{2}$}-- (pi,4);
        \draw [dashed] (2*pi,-4) -- (2*pi,0) node[below left]{$2\pi$} --(2*pi,4);
        \end{axis}
    \end{tikzpicture}
    \end{center}

\clearpage

We can summarise the behaviour of sine, cosine and tangent $\br{\dfrac{\text{sine}}{\text{cosine}}}$ ratios, in each quadrant below:

\begin{center}
    \begin{tikzpicture}[]
        \coordinate[label=below left:$O$] (O) at (0,0);
        \coordinate[] (P) at (1.6069, 1.9151);
        \coordinate[] (XR) at (2.5,0);
        
        \coordinate[label=above right:$1^{\text{st}}$ quadrant (A)] (A) at (1.7, 2);
        \coordinate[label=above left:$2^{\text{nd}}$ quadrant (S)] (B) at (-1.7, 2);
        \coordinate[label=below left:$3^{\text{rd}}$ quadrant (T)] (C) at (-1.7, -2);
        \coordinate[label=below right:$4^{\text{th}}$ quadrant (C)] (D) at (1.7, -2);

        \coordinate[label=below left:$180^{\circ}-\theta$ ] (B1) at (-2.5, 2);
        \coordinate[label=above left:$180^{\circ}+\theta$] (C1) at (-2.5, -2);
        \coordinate[label=above right:$360^{\circ}-\theta$] (D1) at (2.5, -2);
        
        \coordinate[label= above right: 
            $\begin{cases}
            \sin \theta >0 \\
            \cos \theta >0 \\
            \tan \theta >0 \\
        \end{cases}$
        ] (Q1) at (5,1.25);

        \coordinate[label= below right: 
            $\begin{cases}
            \sin \alpha <0 \\
            \textcolor{green!75!black}{\cos \alpha >0} \\
            \tan \alpha <0 \\
        \end{cases}$
        ] (Q4) at (5,-1.25);

\newenvironment{rcases}
  {\left.\begin{aligned}}
  {\end{aligned}\right\rbrace}

        \coordinate[label=above left: 
            $\begin{rcases}
            \textcolor{green!75!black}{\sin \alpha >0} \\
            \cos \alpha <0 \\
            \tan \alpha <0 \\
        \end{rcases}$
        ] (Q2) at (-5,1.25);

        \coordinate[label=below left: 
            $\begin{rcases}
            \sin \alpha <0 \\
            \cos \alpha <0 \\
            \textcolor{green!75!black}{\tan \alpha >0} \\
        \end{rcases}$
        ] (Q3) at (-5,-1.25);
        
        \draw [<-] (0,3) -- (0,-3);
        \draw [->] (-3,0) -- (3,0);
        \draw (0,0) circle (2.5cm);
        \fill (1.6069, 1.9151) circle (2pt);
        \draw (0,0) --  (P);
        \tkzMarkAngle[size=1, ->](XR,O,P); 
        \tkzLabelAngle[pos=0.7](XR,O,P){$\theta$};
        
    \end{tikzpicture}
\end{center}

You should also be familiar with the following exact values:

\begin{center}
    \renewcommand{\arraystretch}{1.75}
    \begin{tabular}{|C{1.5cm}|C{1.5cm}|C{1.5cm}|C{1.5cm}|C{1.5cm}|C{1.5cm}|}
    \hline
    $\theta$ & $0^{\circ}$ & $30^{\circ}$ & $45^{\circ}$ & $60^{\circ}$ & $90^{\circ}$  \\ 
    \hline
    $\sin$ & $0$ & $\dfrac{1}{2}$ & $\dfrac{1}{\sqrt{2}}$ & $\dfrac{\sqrt{3}}{2}$ & $1$  \\ 
    \hline
    $\cos$ & $1$ & $\dfrac{\sqrt{3}}{2}$ & $\dfrac{1}{\sqrt{2}}$ & $\dfrac{1}{2}$ & $0$  \\ 
    \hline
    $\tan$ & $0$ & $\dfrac{1}{\sqrt{3}}$ & $1$ & $\sqrt{3}$ & $\infty$  \\ 
    \hline

    \end{tabular}
    \end{center}


\end{theorybox}

