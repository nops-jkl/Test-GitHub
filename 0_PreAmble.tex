%% PreAmble

\documentclass[a4paper,12pt,addpoints]{exam}
\usepackage{blindtext}
\usepackage[utf8]{inputenc}
\usepackage{color}
\usepackage[left=15mm, right=15mm, top=20mm, bottom=20mm, marginparsep=10mm]{geometry}

\usepackage{amsmath}    % for maths stuff
\usepackage{amssymb}    % for maths stuff
\usepackage{mathabx}    % NESA - HSC Style - Integrals
\usepackage{graphicx}   % for inserting images
\usepackage{tikz}       % for diagrams
\usepackage{tkz-euclide}
\usetikzlibrary{calc}
\usetikzlibrary{patterns,arrows.meta}
\usetikzlibrary{shadows}
\usetikzlibrary{external}
\usetikzlibrary {patterns,patterns.meta}
\usepackage{accents}    % for tilde underneath
\usepackage[most]{tcolorbox}
\usepackage{xcolor}
\usepackage{pst-all}
\usepackage{multicol,multirow}
\usepackage{pgfplots}
\pgfplotsset{compat=newest}
\usepgfplotslibrary{statistics}
\usepgfplotslibrary{fillbetween}
\usepackage{diagbox}
\usepackage{tcolorbox}
\usepackage{multicol}
\usepackage{cancel}
\usepackage{hyperref}
\hypersetup{hidelinks}
\usepackage{esvect} % For arrows above vectors, use \vv{}
\usepackage[thinlines]{easytable}
\usepackage{array}
\newcommand{\PreserveBackslash}[1]{\let\temp=\\#1\let\\=\temp}
\newcolumntype{C}[1]{>{\PreserveBackslash\centering}p{#1}}
\newcolumntype{R}[1]{>{\PreserveBackslash\raggedleft}p{#1}}
\newcolumntype{L}[1]{>{\PreserveBackslash\raggedright}p{#1}}

%% File Details

\title{Trigonometry and Measure of Angles \\ MA-T1}

\author{J. Truong}

%Header and Footer

    \firstpageheader{}{}{}
    \firstpageheadrule

    \runningheader{}{}{}
    \runningheadrule
    
    \firstpagefooter{}{}{}
    \firstpagefootrule

    \runningfooter{J. Truong}{}{Page \thepage}
    \runningfootrule


% Question Formatting
%spacing between questions
\renewcommand{\questionshook}{%
    \setlength{\topsep}{5mm}
    \setlength{\itemsep}{5mm}}

% Spacing between parts
\renewcommand{\partshook}{
    \setlength{\topsep}{5mm}
    \setlength{\itemsep}{5mm}}

% Spacing between subparts
\renewcommand{\subpartshook}{
    \setlength{\topsep}{5mm}
    \setlength{\itemsep}{5mm}}
    \renewcommand{\subpartlabel}{(\thesubpart )}

% Marks Formatting
\marksnotpoints
\pointsinrightmargin
\pointformat{\textbf{\themarginpoints}}





% Writing Space Formatting
\setlength\dottedlinefillheight{8mm}

%%The command \dotlines{X} will produce X dotted lines. 
\newlength\lenlines
\newlength\llines
\newcommand{\dotlines}[1]{
\setlength{\llines}{#1mm}
\setlength{\lenlines}{8\llines+1 mm}
\fillwithdottedlines{\lenlines}
}

% Solution Space Formatting
\renewcommand{\solutiontitle}{\noindent\textbf{Solution:}\par\noindent}


%Brackets
%spacing between questions

%%Shortcuts
\newcommand{\q}{\question}
\newcommand{\p}{\part}
\newcommand{\dpp}{\droppoints}
\newcommand{\dl}{\fillwithdottedlines}
\newcommand{\br}[1]{\left( #1 \right)}
\newcommand{\drv}[3]{ \dfrac{d{#1}}{d{#2}} \left( #3 \right) }
\newcommand{\ddx}[2]{ \dfrac{d}{d{#1}} \left( #2 \right) }


%% BOXES %%

%% THEORY
\newtcolorbox{theorybox}[1]{breakable, colback=red!1!white,
colframe=red!75!white,fonttitle=\bfseries,
title={#1},arc=0mm}

%% Proof
\newtcolorbox{pfbox}[1]{colback=gray!1!white,
colframe=gray!75!white,fonttitle=\bfseries,
title=Proof,arc=0mm}

%% Worked Example
\newtcolorbox[auto counter]{wexample}[2][]{%
colback=orange!1!white,colframe=orange!75!white,fonttitle=\bfseries,
title=Worked example \thetcbcounter: #2,#1,arc=0mm}

%% Example Cloze
\newtcolorbox[auto counter, number within=subsection]{examplecz}[2][]{%
colback=green!1!white,colframe=green!80!black,fonttitle=\bfseries,
title= Example~\thetcbcounter: #2,#1,arc=0mm}

%% Questions
\newtcolorbox[auto counter]{Qbox}[2][]{%
colback=blue!1!white,colframe=blue!70!white,fonttitle=\bfseries,
title= Exercise, breakable,arc=0mm}

%% Answers
\newtcolorbox[auto counter]{Abox}[2][]{%
colback=purple!1!white,colframe=purple!70!white,fonttitle=\bfseries,
title= Answers,arc=0mm}

\renewcommand{\labelenumi}{(\alph{enumi})}
\renewcommand{\labelenumii}{(\roman{enumii})}
\renewcommand{\csc}{\,\text{cosec}\,}
\newcommand{\dint}[0]{\displaystyle \int }

\pgfplotsset{
    standard/.style={
    %axis line style = thick,
    trig format=rad,
    enlargelimits,
    axis x line=middle,
    axis y line=middle,
    enlarge x limits=0.15,
    enlarge y limits=0.15,
    every axis x label/.style={at={(current axis.right of origin)},anchor=north},
    every axis y label/.style={at={(current axis.above origin)},anchor=east},
    axis line style={-{Triangle[length=2.75mm, width=1.5mm]}},
    >={Triangle[length=2.75mm, width=1.5mm]}
    }
}

\counterwithin*{equation}{section}
\counterwithin*{equation}{subsection}


\raggedright

\begin{document}

    \maketitle

    \tableofcontents
    \clearpage

    %\section{Primitive Functions}

\subsection{The Anti-Derivative}

\begin{theorybox}{Notes}
    The sinusoidal functions refer to the sine and cosine functions, $y= \sin x$ and $y=\cos x$ whose graphs are shown below.

    \begin{center}
    \begin{tikzpicture}[>=stealth]
        \begin{axis}[
            standard,
            xlabel = {$x$},
            ylabel = {$y$},
            xmin=-2, xmax=7.7,
            ymin=-1.25, ymax=1.25,
            xtick={1.570796327, 3.141592654, 4.71238898, 6.283185307}, 
            xticklabels={$\frac{\pi}{2}$, $\pi$, $\frac{3\pi}{2}$, $2\pi$}, 
            ytick={-1,1},
            yticklabels={$-1$, $1$},
            height=7cm,
            width=14cm
            ]
        \coordinate[label=below left:$O$] (O) at (0,0);
    
        % Plot 1
        \addplot [samples=1200, domain = 0:6.2831853] { sin(x) };
        \addplot [ dashed, samples=400, domain = -1:0] { sin(x) };
        \addplot [ dashed, samples=400, domain = 6.2831853:7] { sin(x) } node[above]{$y=\sin x$}; 
        
        \draw [blue, dashed, latex-latex] (-1,0) --  node[left, blue]{amplitude} (-1,1) ;
        \draw [blue, dashed, latex-latex] (-1,-1) --  node[left, blue]{amplitude} (-1,0) ;

        \draw [purple, dashed, latex-latex] (0, 1.15) -- node[above, purple]{period} (6.283185307, 1.15);
        \end{axis}
    \end{tikzpicture}
    \end{center}

    \begin{center}
    \begin{tikzpicture}[>=stealth]
        \begin{axis}[
            standard,
            xlabel = {$x$},
            ylabel = {$y$},
            xmin=-2, xmax=7.7,
            ymin=-1.25, ymax=1.25,
            xtick={1.570796327, 3.141592654, 4.71238898, 6.283185307}, 
            xticklabels={$\frac{\pi}{2}$, $\pi$, $\frac{3\pi}{2}$, $2\pi$}, 
            ytick={-1,1},
            yticklabels={$-1$, $1$},
            height=7cm,
            width=14cm
            ]
        \coordinate[label=below left:$O$] (O) at (0,0);
    
        % Plot 1
        \addplot [samples=1200, domain = 0:6.2831853] { cos(x) };
        \addplot [ dashed, samples=400, domain = -1:0] { cos(x) };
        \addplot [ dashed, samples=400, domain = 6.2831853:7] { cos(x) } node[below]{$y=\cos x$}; 

        \draw [blue, dashed, latex-latex] (-1,0) --  node[left, blue]{amplitude} (-1,1) ;
        \draw [blue, dashed, latex-latex] (-1,-1) --  node[left, blue]{amplitude} (-1,0) ;

        \draw [purple, dashed, latex-latex] (0, 1.15) -- node[above, purple]{period} (6.283185307, 1.15);
        \end{axis}
    \end{tikzpicture}
    \end{center}
    Transformations to the functions above can be expressed by writing them in the form
    \begin{center}
        $ y = a \sin \left( b \left(x-c \right) \right) + d $ and $ y = a \cos \left( b \left(x-c \right) \right) + d $
    \end{center}
    where:
    \begin{itemize}
        \item $a$ is the (magnitude of the) amplitude;
        \item the period, $T$, is given by $T= \dfrac{2\pi}{b}$;
        \vskip3mm
        Hint: when drawing sinusoidal graphs, calculate the period, then mark it out using an interval that is divisible by $4$ on the horizontal axis.
        \item $c$ is the phase (horizontal) shift; and
        \item $d$ is the principal (central) axis (or equilibrium).
        
    \end{itemize}
\end{theorybox}

\begin{examplecz}{}
    On separate number planes, sketch the graphs of:
    \begin{enumerate}
    \begin{multicols}{2}
        \item $ y = 2 \sin (x)$
        \item $ y = -3 \cos (x) + 3$
        \item $ y = 3 \sin ( 2x ) $
        \item $ y = 1 - \cos ( 3x ) $
    \end{multicols}
    \end{enumerate}
\end{examplecz}

\vspace*{5mm}

\begin{examplecz}{}
    On separate number planes, sketch the graphs of:
    \begin{enumerate}
    \begin{multicols}{2}
        \item $ y =  \cos \br{2 \br{x- \dfrac{\pi}{4}} } + 1 $
        \item $ y =  - \dfrac{3}{2} \sin \br{ \dfrac{x}{3} + \dfrac{\pi}{2} }$
        \item $ y =  \cos \br{ \pi \br{x+3} } - 2 $
        \item $ y =  3-5\sin \br{ \dfrac{\pi}{2} \br{x+3}} $
    \end{multicols}
    \end{enumerate}
\end{examplecz}

\vspace*{5mm}

\begin{examplecz}{2022 HSC Mathematics Advanced Question 14}
    The graph of $y = k \sin (ax)$ is shown. \hfill \textbf{2}
    \begin{center}
    \begin{tikzpicture}[>=stealth]
        \begin{axis}[
            standard,
            xlabel = {$x$},
            ylabel = {$y$},
            xmin=-25.13274123, xmax=25.13274123,
            ymin=-4, ymax=4,
            xtick={-25.13274123, -18.84955592, -12.56637061, -6.283185307, 6.283185307, 12.56637061, 18.84955592, 25.13274123}, 
            xticklabels={$ - 8 \pi$, $-6\pi$, $-4\pi$, $-2\pi$, $2\pi$, $4\pi$, $6\pi$, $8\pi$}, 
            ytick={-4, -2, 2, 4},
            yticklabels={$-4$, $-2$, $2$, $4$},
            height=6cm,
            width=17.5cm
            ]
        \coordinate[label=below left:$O$] (O) at (0,0);
    
        % Plot 1
        \addplot [samples=2000, domain = -25.13274123:25.13274123] { 4*sin((1/3)*x) };
        \end{axis}
    \end{tikzpicture}
    \end{center}
    What are the values of $a$ and $k$?
\end{examplecz}

\vspace*{5mm}

\begin{examplecz}{}
    Let $ f \br{x} = 5.8 \sin \br{\dfrac{\pi}{6} \br{x+1}} +b $. The graph of $y=f\br{x}$ has a local maximum at $ \br{2,\,21.8}$ and a local minimum at $ \br{8,\,10.2}$.
    \begin{enumerate}
        \item Find the period of $f$.
        \item Find the value of $b$.
    \end{enumerate}
\end{examplecz}

\begin{examplecz}{}
    Let $ f \br{x} = p \sin \br{\dfrac{2\pi}{9} \br{x-3.75}} +q $. The graph of $y=f\br{x}$ passes through the points $\br{3,\,2.5}$ and $6,\,15.1$.
    \vskip3mm
    Find the values of $p$ and $q$.
\end{examplecz}

\clearpage

\subsection{Boundary Conditions}

\begin{theorybox}{Notes}
    The graph of $y= \tan x$ is shown below.

    \begin{center}
    \begin{tikzpicture}[>=stealth]
        \begin{axis}[
            standard,
            xlabel = {$x$},
            ylabel = {$y$},
            xmin=-3.1, xmax=6.3,
            ymin=-4, ymax=4,
            xtick={-3.141592654, -1.570796327, 0.7853981634, 1.570796327, 3.141592654, 4.71238898, 6.283185307}, 
            xticklabels={$-\pi$,$-\frac{\pi}{2}$,$\frac{\pi}{4}$, $\frac{\pi}{2}$, $\pi$, $\frac{3\pi}{2}$, $2\pi$}, 
            ytick={1},
            yticklabels={$1$},
            height=10cm,
            width=16cm
            ]
        \coordinate[label=below left:$O$] (O) at (0,0);
    
        % Plot 1
        \addplot [samples=800, domain = -3.141592654:-1.81577] { tan(x) };
        \addplot [samples=800, domain = -1.32582:1.32582] { tan(x) };
        \addplot [samples=800, domain = 1.81577:4.46741] { tan(x) };
        \addplot [samples=800, domain = 4.95737:6.283185307] { tan(x) };

        \draw [dashed] (-1.570796327, -4) -- (-1.570796327, 4);
        \draw [dashed] (1.570796327, -4) -- (1.570796327, 4);
        \draw [dashed] (4.71238898, -4) -- (4.71238898, 4);

        \fill (axis cs: 0.7853981634,1) circle (2pt);

        \draw [purple, dashed, latex-latex] (-1.570796327, -4.15) -- node[below, purple]{period} (1.570796327, -4.15);
        \end{axis}
    \end{tikzpicture}
    \end{center}

    Transformations to the functions above can be expressed by writing them in the form
    \begin{center}
        $ y = a \tan \left( b \left(x-c \right) \right) + d $
    \end{center}
    where:
    \begin{itemize}
        \item the period, $T$, is given by $T= \dfrac{\pi}{b}$; and
        \item $c$ is the phase (horizontal) shift.        
    \end{itemize}
\end{theorybox}

\begin{examplecz}{}
    On separate number planes, sketch the graphs of:
    \begin{enumerate}
    \begin{multicols}{3}
        \item $ y =  \tan \br{2x} $
        \item $ y =  - \tan \br{ \dfrac{x}{2} }$
        \item $ y =  \tan \br{ 2 \br{x - \dfrac{\pi}{4}} } $
    \end{multicols}
    \end{enumerate}
\end{examplecz}



%\vfill

%\begin{Qbox}{Homework}
%    From CambridgeMATHS - Mathematics Extension 1
%    \begin{itemize}
%    \begin{multicols}{2}
%        \item Exercise 6C: 1 to 3, 6, 8, 10, 11 and 13
%    \end{multicols}
%    \end{itemize}
    
%\end{Qbox}

\clearpage
    \section{Trigonometry with non-right-angled triangles}
\subsection{The Sine Rule}
\begin{theorybox}{Notes}

The vertices (corners) of any figure are labelled using capital letters. In a triangle, the sides are labelled using the lower case letter of the opposing vertex.

\begin{center}
    \begin{tikzpicture}[scale=0.65]
        \draw (0,0) node[below left]{$A$} -- node[below right]{$c$}(6,2)node[right]{$B$} -- node[above right]{$a$} (2,5) node[above]{$C$} -- node[left]{$b$}cycle;
    \end{tikzpicture}
\end{center}

The sine rule states that $\dfrac{a}{\sin A}=\dfrac{b}{\sin B}=\dfrac{a}{\sin C}$.
\vskip5mm
When finding an unknown angle, we can use the form $\dfrac{\sin A}{a}=\dfrac{\sin B}{b}=\dfrac{\sin C}{c}$.

\end{theorybox}

\begin{pfbox}{Proof}

The diagram below shows triangle $ABC$ where $AC=b$, $BC=a$ and point $M$ lies on $AB$ such that $AB \perp CM$.
\begin{center}
    \begin{tikzpicture}
        \draw (0,0) node[below left]{$A$} -- (7,0)node[right]{$B$} -- node[above right]{$a$} (4,3) node[above]{$C$} -- node[above left]{$b$}cycle;
        \draw[dashed] (4,3) -- (4,0) node[below]{$M$};
        \draw (3.75,0) -- (3.75,0.25) -- (4,0.25);
    \end{tikzpicture}
\end{center}

From $\triangle ACM$, $\sin A = \dfrac{CM}{b} \Rightarrow CM = b \sin A$.
\vskip5mm
Similarly in $\triangle BCM$, $\sin B = \dfrac{CM}{a} \Rightarrow CM = a \sin B$.
\vskip5mm
Equating these expressions for $CM$ gives:
\begin{align*}
    b \sin A &= a \sin B \\
    \therefore \dfrac{a}{\sin A}&=\dfrac{b}{\sin B}
\end{align*}
This process can be repeated involving vertex $C$ and its opposing side to give the sine rule as it is conventionally stated.
\end{pfbox}


%% Worked Examples

\begin{examplecz}{Finding an unknown side}

In $\triangle ABC$, $AC=21$\,cm, $BC=x$\,cm, $\angle CAB=38^{\circ}$ and $\angle CBA=43$ as shown.
\begin{center}
    \begin{tikzpicture}
        \coordinate[label=left: $A$] (A) at (0,0);
        \coordinate[label=right: $B$] (B) at (6.5,-1);
        \coordinate[label=above: $C$] (C) at (4,2);
        \draw (A) -- (B) -- node[above right]{$x$\,cm} (C) -- node[above left]{$21$\,cm}cycle;
        \tkzLabelAngle[pos=1.2](C,B,A){$43^{\circ}$};
        \tkzLabelAngle[pos=1.2](B,A,C){$38^{\circ}$};        
    \end{tikzpicture}
\end{center}
Find the value of $x$, correct to two decimal places.
\tcblower
\textbf{Solution:}
\vspace*{40mm}
\begin{comment}
\begin{align*}
    \dfrac{x}{\sin 38^{\circ}} &= \dfrac{21}{\sin 43^{\circ}} \\
    x &= \dfrac{21\times \sin 38^{\circ}}{\sin 43^{\circ}} \\
    \therefore x &= 18.96\,\text{cm}
\end{align*}
\end{comment}

\end{examplecz}

\vspace*{5mm}


\begin{examplecz}{Finding an unknown angle}

In $\triangle ABC$ shown, $AB=10\,\text{cm}$, $BC=7\,\text{cm}$ and $\angle ACB=110^{\circ}$ and $\angle BAC=\theta$.
\begin{center}
    \begin{tikzpicture}
        \coordinate[label=below left: $A$] (A) at (0,0);
        \coordinate[label=right: $B$] (B) at (6.5,1.5);
        \coordinate[label=above: $C$] (C) at (2,3);
        \draw (A) -- node[below]{$10$\,cm}(B) -- node[above right]{$7$\,cm} (C) -- cycle;
        \tkzLabelAngle[pos=0.7](A,C,B){$110^{\circ}$};
        \tkzLabelAngle[pos=1](B,A,C){$\theta$};        
    \end{tikzpicture}
\end{center}
Find the value of $\theta$, correct to two decimal places.
\tcblower
\textbf{Solution:}
\vspace*{40mm}
\begin{comment}
\begin{align*}
    \dfrac{\sin \theta}{7} &= \dfrac{\sin 110 ^{\circ}}{10} \\
    \sin \theta &= \dfrac{7 \sin 110^{\circ}}{10} \\
    \therefore \theta &= \sin ^{-1} \left(\dfrac{7 \sin 110^{\circ}}{10}  \right) \\
    &= 41.13^{\circ}
\end{align*}
\end{comment}

\end{examplecz}

%% Self-paced Questions 

\begin{examplecz}{}

Triangle $HIJ$ has sides $IJ=14$\,cm and $HJ=x$\,cm. $\angle JHI=37^{\circ}$ and $\angle HIJ=52^{\circ}$ as shown. 
\begin{center}
    \begin{tikzpicture}
        \coordinate[label=left: $H$] (H) at (0,0);
        \coordinate[label=right: $I$] (I) at (6,0);
        \coordinate[label=below: $J$] (J) at (4,-2.5);
        \draw (H) -- (I) -- node[below right]{$14$\,cm} (J) -- node[below left]{$x$\,cm} cycle;
        \tkzLabelAngle[pos=1.4](J,H,I){$37^{\circ}$};
        \tkzLabelAngle[pos=1.1](H,I,J){$52^{\circ}$};        
    \end{tikzpicture}
\end{center}
Find the value of $x$, correct to one decimal place.
\tcblower
\textbf{Solution:}
\vspace*{4.5cm}
%\begin{align*}
%    \dfrac{\sin \theta}{13} &= \dfrac{\sin 89 ^{\circ}}{25} \\
%    \sin \theta &= \dfrac{13 \sin 89^{\circ}}{25} \\
%    \therefore \theta &= \sin ^{-1} \left(\dfrac{13 \sin 89^{\circ}}{25}  \right) \\
%    &= 31^{\circ}
%\end{align*}
\end{examplecz}

\vspace{5mm}

\begin{examplecz}{}
In $\triangle PQR$ shown, $PQ=25$\,cm, $PR=13$\,cm, $\angle PRQ=89^{\circ}$ and $\angle PQR=\theta$.
\begin{center}
    \begin{tikzpicture}
        \coordinate[label=left: $P$] (P) at (0,0);
        \coordinate[label=right: $Q$] (Q) at (6,0);
        \coordinate[label=above: $R$] (R) at (2,3);
        \draw (P) -- node[below]{$25$\,cm}(Q) -- (R) -- node[above left]{$13$\,cm} cycle;
        \tkzLabelAngle[pos=0.7](P,R,Q){$89^{\circ}$};
        \tkzLabelAngle[pos=1](R,Q,P){$\theta$};        
    \end{tikzpicture}
\end{center}
Find the value of $\theta$, correct to the nearest degree.
\tcblower
\textbf{Solution:}
\vspace*{4.5cm}
%\begin{align*}
%    \dfrac{\sin \theta}{13} &= \dfrac{\sin 89 ^{\circ}}{25} \\
%    \sin \theta &= \dfrac{13 \sin 89^{\circ}}{25} \\
%    \therefore \theta &= \sin ^{-1} \left(\dfrac{13 \sin 89^{\circ}}{25}  \right) \\
%    &= 31^{\circ}
%\end{align*}
\end{examplecz}

\begin{examplecz}{}
In $\triangle XYZ$, $XY=20.8$\,cm, $YZ=17.6$\,cm, $\angle XZY=26^{\circ}$ and $\angle XYZ=\theta$ as shown.
\begin{center}
    \begin{tikzpicture}
        \coordinate[label=left: $X$] (X) at (0,0);
        \coordinate[label=right: $Y$] (Y) at (5,0);
        \coordinate[label=above: $Z$] (Z) at (7,3);
        \draw (X) -- node[below]{$20.8$\,cm}(Y) -- node[below right]{$17.6$\,cm} (Z) -- cycle;
        \tkzLabelAngle[pos=1.3](X,Z,Y){$26^{\circ}$};
        \tkzLabelAngle[pos=0.5](Z,Y,X){$\theta$};        
    \end{tikzpicture}
\end{center}
Find the value of $\theta$, correct to the nearest minute.
\tcblower
\textbf{Solution:}
\vspace*{5cm}
%\begin{align*}
%    \dfrac{\sin \theta}{13} &= \dfrac{\sin 89 ^{\circ}}{25} \\
%    \sin \theta &= \dfrac{13 \sin 89^{\circ}}{25} \\
%    \therefore \theta &= \sin ^{-1} \left(\dfrac{13 \sin 89^{\circ}}{25}  \right) \\
%    &= 31^{\circ}
%\end{align*}
\end{examplecz}




\vfill
\begin{Qbox}{}
    From \textit{CambridgeMaths Year 11 Mathematics Extension 1}:
    \begin{itemize}
        \item Exercise 6I: 1, 2, 3, 5 and 12
    \end{itemize}
\end{Qbox}
\clearpage
\subsection{The Cosine Rule}

\begin{theorybox}{Notes}

The vertices of a triangle are labelled with capital letters, and the sides opposite each vertex are labelled with the corresponding lower-case letter.

\begin{center}
    \begin{tikzpicture}[scale=0.65]
        \draw (0,0) node[below left]{$A$} -- node[below right]{$c$}(6,2)node[right]{$B$} -- node[above right]{$a$} (2,5) node[above]{$C$} -- node[left]{$b$}cycle;
    \end{tikzpicture}
\end{center}

The Cosine Rule states:
$$ a^2 = b^2 + c^2 - 2bc \cos A $$
Similarly,
$$ b^2 = a^2 + c^2 - 2ac \cos B $$
and
$$ c^2 = a^2 + b^2 - 2ab \cos C. $$

To find an unknown angle slightly easier, the cosine rule can be re-arranged to give:

$$ \cos A = \dfrac{b^2 + c^2 - a^2}{2bc}$$

\end{theorybox}

\begin{pfbox}{Proof}

The diagram below shows triangle $ABC$ with side lengths $AC=b$, $BC=a$, and $AB=c$. We draw an altitude from $C$ to side $AB$ at point $D$.

\begin{center}
    \begin{tikzpicture}
        \coordinate[label=left:$A$] (A) at (0,0);
        \coordinate[label=right:$B$] (B) at (6,0);
        \coordinate[label=above:$C$] (C) at (4,3);
        \coordinate[label=below:$D$] (D) at (4,0);
        \draw (A) -- (B) -- node[above right]{$a$} (C) -- node[above left]{$b$} cycle;
        \draw[dashed] (C) -- node[right]{$h$} (D);
        \draw (3.8,0) -- (3.8,0.2) -- (4,0.2);
        \node at (2.2,-0.3) {$x$};
        \node at (5,-0.3) {$c-x$};
    \end{tikzpicture}
\end{center}

From right-angled triangle $ADC$:
\[
\cos A = \frac{x}{b} \quad \Rightarrow \quad x = b \cos A
\]
and
\[
h^2 = b^2 - x^2.
\]

From right-angled triangle $BDC$:
\[
a^2 = h^2 + (c-x)^2.
\]

Substituting for $h^2$:
\begin{align*}
    a^2 &= \left(b^2 - x^2\right) + (c-x)^2 \\
        &= b^2 - x^2 + c^2 - 2cx + x^2 \\
        &= b^2 + c^2 - 2cx.
\end{align*}

Substituting $x = b \cos A$ gives the cosine rule.:
\[
a^2 = b^2 + c^2 - 2bc \cos A.
\]
\end{pfbox}

\newpage

%% Worked Examples

\begin{examplecz}{Finding an unknown side}

In $\triangle ABC$, $AB=8$\,cm, $AC=6$\,cm and $\angle BAC=60^\circ$ as shown.
\begin{center}
    \begin{tikzpicture}[scale=0.9]
        \coordinate[label=left:$A$] (A) at (0,0);
        \coordinate[label=right:$B$] (B) at (6,0);
        \coordinate[label=above:$C$] (C) at (2.5,3.5);
        \draw (A) -- node[below]{$8$\,cm} (B) -- (C) -- node[left]{$6$\,cm} (A) -- cycle;
        \tkzLabelAngle[pos=1](B,A,C){$60^\circ$};
    \end{tikzpicture}
\end{center}
Find the length of $BC$, correct to two decimal places.
\tcblower
\textbf{Solution:}
\vspace*{40mm}
\begin{comment}
\begin{align*}
    BC^2 &= 6^2 + 8^2 - 2(6)(8)\cos 60^\circ \\
    &= 36 + 64 - 96(0.5) \\
    &= 100 - 48 \\
    &= 52 \\
    \therefore BC &= \sqrt{52} \approx 7.21\,\text{cm}.
\end{align*}
\end{comment}
\end{examplecz}

\vspace{3mm}

\begin{examplecz}{Finding an unknown angle}

In $\triangle XYZ$, $XY=9$\,cm, $YZ=11$\,cm and $XZ=14$\,cm.
\begin{center}
    \begin{tikzpicture}[scale=0.9]
        \coordinate[label=left:$X$] (X) at (0,0);
        \coordinate[label=right:$Y$] (Y) at (6,0);
        \coordinate[label=above:$Z$] (Z) at (3,4);
        \draw (X) -- node[below]{$9$\,cm} (Y) -- node[right]{$11$\,cm} (Z) -- node[above left]{$14$\,cm} (X) -- cycle;
        \tkzLabelAngle[pos=1](Y,X,Z){$\theta$};
    \end{tikzpicture}
\end{center}
Find the value of $\theta$, correct to one decimal place.
\tcblower
\textbf{Solution:}
\vspace*{40mm}
\begin{comment}
\begin{align*}
    11^2 &= 9^2 + 14^2 - 2(9)(14)\cos \theta \\
    121 &= 81 + 196 - 252 \cos \theta \\
    121 &= 277 - 252 \cos \theta \\
    252 \cos \theta &= 277 - 121 \\
    252 \cos \theta &= 156 \\
    \cos \theta &= \dfrac{156}{252} \\
    &= \dfrac{13}{21} \\
    \therefore \theta &= \cos^{-1}\left(\dfrac{13}{21}\right) \\
    &\approx 51.1^\circ
\end{align*}
\end{comment}
\end{examplecz}

\vspace{5mm}

%% Self-paced Questions

\begin{examplecz}{}

In $\triangle DEF$, $DE=7.2$\,cm, $DF=5.5$\,cm and $\angle EDF=47^\circ$.
\begin{center}
    \begin{tikzpicture}
        \coordinate[label=left:$D$] (D) at (0,0);
        \coordinate[label=right:$E$] (E) at (6,0);
        \coordinate[label=above:$F$] (F) at (3,3);
        \draw (D) -- node[below]{$7.2$\,cm} (E) -- (F) -- node[above left]{$5.5$\,cm} (D) -- cycle;
        \tkzLabelAngle[pos=1](E,D,F){$47^\circ$};
    \end{tikzpicture}
\end{center}
Find the length of $EF$, correct to one decimal place.
\tcblower
\textbf{Solution:}
\vspace*{4.25cm}
\end{examplecz}

\vspace{3mm}

\begin{examplecz}{}

In $\triangle GHI$, $GH=12$\,cm, $HI=9$\,cm and $GI=10$\,cm.
\begin{center}
    \begin{tikzpicture}
        \coordinate[label=left:$G$] (G) at (0,0);
        \coordinate[label=right:$H$] (H) at (5,0);
        \coordinate[label=above:$I$] (I) at (2,3);
        \draw (G) -- node[below]{$12$\,cm} (H) -- node[above right]{$9$\,cm} (I) -- node[above left]{$10$\,cm} (G) -- cycle;
        \tkzLabelAngle[pos=1](H,G,I){$\theta$};
    \end{tikzpicture}
\end{center}
Find the size of angle $\theta$, correct to the nearest degree.
\tcblower
\textbf{Solution:}
\vspace*{4.25cm}
\end{examplecz}

\newpage

\begin{examplecz}{}
Two jetskis depart from point $P$ at the same time. Jetski $A$ travels $3.6$\,km north and jetski $B$ travels $6$\,km on a bearing of $028^{\circ}$.
\vskip3mm
Using only the cosine rule, find the distance (to two decimal places), and bearing (to the nearest degree) of jetski $B$ from jetski $A$.
\tcblower
\textbf{Solution:}
\vspace{10cm}
\begin{comment}
\begin{center}
    \begin{tikzpicture}
        \coordinate[label=left: $P$] (P) at (0,0);
        \coordinate[label=above left: $A$] (A) at (0,3.6);
        \coordinate[label=above right: $B$] (B) at (2.82,5.3);
        \draw (P) -- node[left]{$3.6$\,km}(A) -- (B) -- node[below right]{$6$\,km} cycle;
        \tkzLabelAngle[pos=1.4](B,P,A){$28^{\circ}$};        
    \end{tikzpicture}
\end{center}
\end{comment}
\end{examplecz}

\vfill

\begin{Qbox}{}
    From \textit{CambridgeMaths Year 11 Mathematics Extension 1}:
    \begin{itemize}
        \item Exercise 6J: 1 to 7, 11
    \end{itemize}
\end{Qbox}
\newpage
\subsection{The area of a triangle}

\begin{theorybox}{Notes}

The area, $A$, of any triangle can be calculated using the formula $$A=\dfrac{1}{2}ab\sin C$$
\end{theorybox}

\begin{pfbox}{Proof}

The diagram below shows triangle $ABC$ where $AC=b$, $BC=a$ and point $M$ lies on $AB$ such that $AM \perp BC$.
\begin{center}
    \begin{tikzpicture}
        \draw (0,0) node[below left]{$C$} -- (7,0)node[below right]{$B$} -- node[above right]{$c$} (4,3) node[above]{$A$} -- node[above left]{$b$}cycle;
        \draw[dashed] (4,3) -- (4,0) node[below]{$M$};
        \draw[dashed, stealth-stealth] (0,-0.75) -- node[below]{$a$} (7,-0.75);
        \draw (3.75,0) -- (3.75,0.25) -- (4,0.25);
    \end{tikzpicture}
\end{center}

From $\triangle ACM$, $\sin C = \dfrac{AM}{b} \Rightarrow AM = b \sin C$.
\vskip5mm
Now, using the fact that the area, $A$ of the triangle is
$$\dfrac{1}{2} \times \text{base length} \times \text{perpendicular height}$$
We get
\begin{align*}
    A &= \dfrac{1}{2} \times a \times b \sin C \\
    \therefore A &= \dfrac{1}{2} a b \sin C
\end{align*}
This process can be repeated involving vertices $A$ and $B$ to give other variations of the formula.
\end{pfbox}

\vfill

\begin{Qbox}{}
    From \textit{CambridgeMaths Year 11 Mathematics Extension 1}:
    \begin{itemize}
        \item Exercise 6I: 4, 6, 13, 14, 16
    \end{itemize}
\end{Qbox}


\newpage


\begin{examplecz}{}

In triangle $ABC$ below, $AB=15$\,cm, $BC=9$\,cm and $\angle ABC=43^{\circ}$. 
\begin{center}
    \begin{tikzpicture}
        \coordinate[label=left: $A$] (A) at (0,0);
        \coordinate[label=right: $B$] (B) at (6,0);
        \coordinate[label=above: $C$] (C) at (3.5, 2.5);
        \draw (A) -- node[below]{$15$\,cm} (B) -- node[above right]{$9$\,cm} (C) --  cycle;
        \tkzLabelAngle[pos=1.2](C,B,A){$43^{\circ}$};
        %\tkzLabelAngle[pos=1.1](H,I,J){$52^{\circ}$};        
    \end{tikzpicture}
\end{center}
Find the area of triangle $ABC$, correct to two decimal places.
\tcblower
\textbf{Solution:}
\vspace*{4cm}
%\begin{align*}
%    \dfrac{\sin \theta}{13} &= \dfrac{\sin 89 ^{\circ}}{25} \\
%    \sin \theta &= \dfrac{13 \sin 89^{\circ}}{25} \\
%    \therefore \theta &= \sin ^{-1} \left(\dfrac{13 \sin 89^{\circ}}{25}  \right) \\
%    &= 31^{\circ}
%\end{align*}
\end{examplecz}

\vspace{5mm}

\begin{comment}
\begin{examplecz}{}

Triangle $ABC$ has sides $AC=21.4$\,cm, $BC=13.5$\,cm and $\angle ACB=76^{\circ}$ as shown. 
\begin{center}
    \begin{tikzpicture}[scale=0.85]
        \coordinate[label=left: $A$] (A) at (0,-2);
        \coordinate[label=right: $B$] (B) at (6,0);
        \coordinate[label=above: $C$] (C) at (2.5, 3);
        \draw (A) -- (B) -- node[above right]{$13.5$\,cm} (C) -- node[above left]{$21.4$\,cm}  cycle;
        \tkzLabelAngle[pos=0.8](A,C,B){$76^{\circ}$};
        %\tkzLabelAngle[pos=1.1](H,I,J){$52^{\circ}$};        
    \end{tikzpicture}
\end{center}
Find the area of triangle $ABC$, correct to two decimal places.
\tcblower
\textbf{Solution:}
\vspace*{4cm}
%\begin{align*}
%    \dfrac{\sin \theta}{13} &= \dfrac{\sin 89 ^{\circ}}{25} \\
%    \sin \theta &= \dfrac{13 \sin 89^{\circ}}{25} \\
%    \therefore \theta &= \sin ^{-1} \left(\dfrac{13 \sin 89^{\circ}}{25}  \right) \\
%    &= 31^{\circ}
%\end{align*}
\end{examplecz}
\end{comment}

\begin{examplecz}{}
Triangle $PQR$ has an area of $155\,\text{cm}^2$. $PQ=20$\,cm, $QR=17.5$\,cm and $\angle PQR=\theta$ as shown. 
\begin{center}
    \begin{tikzpicture}[scale=0.75]
        \coordinate[label=above left: $P$] (P) at (0,5);
        \coordinate[label=below left: $Q$] (Q) at (0,0);
        \coordinate[label=right: $R$] (R) at (4, 2.5);
        \draw (P) -- node[left]{$20$\,cm} (Q) -- node[below right]{$17.5$\,cm}  (R) --  cycle;
        \tkzLabelAngle[pos=0.8](R,Q,P){$\theta$};
        %\tkzLabelAngle[pos=1.1](H,I,J){$52^{\circ}$};        
    \end{tikzpicture}
\end{center}
Find the value of  $\theta$, correct to the nearest degree.
\tcblower
\textbf{Solution:}
\vspace*{4cm}
%\begin{align*}
%    \dfrac{\sin \theta}{13} &= \dfrac{\sin 89 ^{\circ}}{25} \\
%    \sin \theta &= \dfrac{13 \sin 89^{\circ}}{25} \\
%    \therefore \theta &= \sin ^{-1} \left(\dfrac{13 \sin 89^{\circ}}{25}  \right) \\
%    &= 31^{\circ}
%\end{align*}
\end{examplecz}





\newpage
\subsection{Problems involving non-right-angled triangles}

\begin{examplecz}{}
Three sides of a triangle measure $7$\,cm, $9$\,cm and $12$\,cm respectivey.
\vskip3mm
Find the area of the triangle, correct to one decimal place.
\tcblower
\textbf{Solution:}
\vspace*{6.5cm}
\end{examplecz}

\vspace{5mm}

\begin{examplecz}{}
In triangle $AMN$ shown, $AM=15$\,cm, $MN=12.3$\,cm and $\angle NAM=42^{\circ}$ as shown.
\begin{center}
    \begin{tikzpicture}
        \coordinate[label=below left: $A$] (A) at (0,0);
        \coordinate[label=below right: $M$] (M) at (6,0);
        \coordinate[label=above right: $N$] (N) at (5,3.5);
        
        \draw (A) -- node[below]{$15$\,cm} (M) -- node[above right]{$12.3$\,cm}  (N) --  cycle;
        \tkzLabelAngle[pos=1.2](M,A,N){$42^{\circ}$};
        %\tkzLabelAngle[pos=1.1](H,I,J){$52^{\circ}$};        
    \end{tikzpicture}
\end{center}
Find the area of the triangle, correct to two decimal places.
\tcblower
\textbf{Solution:}
\vspace*{4.5cm}
%\begin{align*}
%    \dfrac{\sin \theta}{13} &= \dfrac{\sin 89 ^{\circ}}{25} \\
%    \sin \theta &= \dfrac{13 \sin 89^{\circ}}{25} \\
%    \therefore \theta &= \sin ^{-1} \left(\dfrac{13 \sin 89^{\circ}}{25}  \right) \\
%    &= 31^{\circ}
%\end{align*}
\end{examplecz}

\vspace{5mm}

\begin{examplecz}{}
In an acute angled triangle $ABC$, $b=14$\,cm, $c=16$\,cm and $\cos A= \dfrac{7}{10}$.
\vskip3mm
Find the exact area of the triangle.
\tcblower
\textbf{Solution:}
\vspace*{7.5cm}
%\begin{align*}
%    \dfrac{\sin \theta}{13} &= \dfrac{\sin 89 ^{\circ}}{25} \\
%    \sin \theta &= \dfrac{13 \sin 89^{\circ}}{25} \\
%    \therefore \theta &= \sin ^{-1} \left(\dfrac{13 \sin 89^{\circ}}{25}  \right) \\
%    &= 31^{\circ}
%\end{align*}
\end{examplecz}

\newpage

\begin{examplecz}[height fill=true]{}
Two surveying drones depart from point $P$ at the same time. Drone $A$ flies on a bearing of $280^{\circ}$ at $60$\,m/s and drone $B $ flies on a bearing of $015^{\circ}$ at $50$\,m/s. After $2$ minutes, the drones stop and hold their position.
\vskip3mm
\begin{enumerate}
    \item [(a)] Find the distance between the drones when they stop.
    \item  [(b)] Find the bearing from drone $A$ to drone $B$ after $2$ minutes.
    \item [(c)] Find the area enclosed by the edges joining point $P$ and the locations of the drones after $2$ minutes.  
\end{enumerate}
\tcblower
\textbf{Solution:}
\vspace{2cm}
%\begin{align*}
%    \dfrac{\sin \theta}{13} &= \dfrac{\sin 89 ^{\circ}}{25} \\
%    \sin \theta &= \dfrac{13 \sin 89^{\circ}}{25} \\
%    \therefore \theta &= \sin ^{-1} \left(\dfrac{13 \sin 89^{\circ}}{25}  \right) \\
%    &= 31^{\circ}
%\end{align*}
\end{examplecz}

\newpage

\begin{examplecz}[height fill=true]{}

Let points $P$ and $Q$ denote the top and bottom of a tower respectively. Point $A$ is due east of a tower. From another point $B$, the tower is on a bearing of $051^{\circ}$. The angles of elevation to $P$ from $A$ and $B$ are $12^{\circ}$ and $11^{\circ}$. Points $A$ and $B$ are $1000$ metres apart as shown.
\begin{center}
    \begin{tikzpicture}
        \coordinate[label = below: $Q$] (Q) at (0,0);
        \coordinate[label = above: $P$] (P) at (0,3);
        \coordinate[label = left: $B$] (B) at (-3,-2);
        \coordinate[label = right: $A$] (A) at (5,0);

        \draw (B) -- (P) -- (A) -- node[below right]{$1000$\,m} cycle;
        \draw[dashed] (P) -- node[right]{$h$} (Q);
        \draw[dashed] (B) -- (Q) -- (A);

        \tkzLabelAngle[pos=1.4](Q,B,P){$11^{\circ}$};
        \tkzLabelAngle[pos=1.4](P,A,Q){$12^{\circ}$};
        \tkzMarkRightAngle[size=0.35](P,Q,B);
        \tkzMarkRightAngle[size=0.35](A,Q,P);
        %\tkzMarkAngle[size=0.75cm,color=cyan,label=$\theta$](B,P,A);        
    \end{tikzpicture}
\end{center}
\begin{enumerate}
    \item Show that $\angle AQB = 141^{\circ}$.
    \item Using $\Delta APQ$, show that $AQ=h\tan 78^{\circ}$, where $h$ is the height of the tower.
    \item Hence, find the value of $h$.
\end{enumerate}

\tcblower
\textbf{Solution:}
%\vspace*{12cm}
%\begin{align*}
%    \dfrac{\sin \theta}{13} &= \dfrac{\sin 89 ^{\circ}}{25} \\
%    \sin \theta &= \dfrac{13 \sin 89^{\circ}}{25} \\
%    \therefore \theta &= \sin ^{-1} \left(\dfrac{13 \sin 89^{\circ}}{25}  \right) \\
%    &= 31^{\circ}
%\end{align*}
\end{examplecz}


\clearpage
    \newpage
\section{Angles of any magnitude}

\subsection{The unit circle}

\begin{theorybox}{}

Consider the a circle of radius $r$ centred at the origin.
Points on the circle can be described using in two ways:
    \begin{itemize}
        \item Cartesian coordinates $\br{x,\,y}$, and
        \item \textit{polar form} which uses:
            \begin{itemize}
                \item the distance between the point and the origin and,
                \item the size of the angle between the line connecting the point to the origin and the positive $x$-axis.
            \end{itemize}
    \end{itemize}  
\begin{center}
    \begin{tikzpicture}[>={Triangle[length=2.75mm, width=1.5mm]}]
        \coordinate[label=below left:$O$] (O) at (0,0);
        \coordinate[] (P) at (1.6069, 1.9151);
        \coordinate[label= below right: $r$] (XR) at (2.5,0);
        
        \draw [<-] (0,3) -- (0,-3);
        \draw [->] (-3,0) -- (3,0);
        \draw (0,0) circle (2.5cm);
        \fill (1.6069, 1.9151) circle (2pt);
        \draw (0,0) -- node[above left]{$r$} (P) node[above right]{$P \left(x,\,y\right)$};
        \tkzMarkAngle[size=1, -latex](XR,O,P); 
        \tkzLabelAngle[pos=0.7](XR,O,P){$\theta$};
        \draw [dashed, latex-latex] (1.6069,0) -- node[right]{$y$} (1.6069, 1.8);
        \draw [dashed, latex-latex] (0,-0.1) -- node[below]{$x$} (1.6069, -0.1);
    \end{tikzpicture}
\end{center}

By equating these two forms, we can see that:

\begin{itemize}
    \item the cosine ratio gives
        \begin{align*}
        \cos \theta &= \dfrac{x}{r} \\
        \therefore y &= \cos \theta
        \end{align*}

    \item the sine ratio gives
        \begin{align*}
        \sin \theta &= \dfrac{y}{r} \\
        \therefore y &= \sin \theta
        \end{align*}
\end{itemize}

Since these are ratios, we can often let the radius of the circle be $1$~unit and say that:
\begin{itemize}
    \item $\cos \theta$ refers to the horizontal coordinate of the any point on the unit circle, and
    \item $\sin \theta$ refers to the vertical coordinate of the any point on the unit circle.
\end{itemize}

  \newpage


For interactive applets showing how the coordinates of the point as it moves around the circle, are related to the cosine and sine graphs, visit:
\begin{itemize}
    \item \url{https://www.geogebra.org/m/pcf9wwwa} and 
    \item \url{https://www.geogebra.org/m/qugk8rk5} 
\end{itemize}
The diagrams below show the values of $\cos \theta$ ($x$-coordinate) and $\sin \theta$ ($y$-coordinate) of a point as it completes one revolution of a unit circle.
\begin{center}
    \begin{tikzpicture}[>=stealth]
        \begin{axis}[
            standard,
            xlabel = {$x$},
            ylabel = {$y$},
            xmin=-2, xmax=7.7,
            ymin=-1, ymax=1,
            xtick={1.570796327, 3.141592654, 4.71238898, 6.283185307}, 
            xticklabels={$90^{\circ}$, $180^{\circ}$, $270^{\circ}$, $360^{\circ}$}, 
            ytick={-1,1},
            yticklabels={$-1$, $1$},
            height=6cm,
            width=12cm
            ]
        \coordinate[label=below left:$O$] (O) at (0,0);
    
        % Plot 1
        \addplot [samples=1200, domain = 0:6.2831853] { sin(x) };
        \addplot [ dashed, samples=400, domain = -1:0] { sin(x) };
        \addplot [ dashed, samples=400, domain = 6.2831853:7] { sin(x) } node[above]{$y=\sin x$}; 
        
        %\draw [blue, dashed, latex-latex] (-1,0) --  node[left, blue]{amplitude} (-1,1) ;
        %\draw [blue, dashed, latex-latex] (-1,-1) --  node[left, blue]{amplitude} (-1,0) ;

        %\draw [purple, dashed, latex-latex] (0, 1.15) -- node[above, purple]{period} (6.283185307, 1.15);
        \end{axis}
    \end{tikzpicture}
    \end{center}

    \begin{center}
    \begin{tikzpicture}[>=stealth]
        \begin{axis}[
            standard,
            xlabel = {$x$},
            ylabel = {$y$},
            xmin=-2, xmax=7.7,
            ymin=-1, ymax=1,
            xtick={1.570796327, 3.141592654, 4.71238898, 6.283185307}, 
            xticklabels={$90^{\circ}$, $180^{\circ}$, $270^{\circ}$, $360^{\circ}$}, 
            ytick={-1,1},
            yticklabels={$-1$, $1$},
            height=6cm,
            width=12cm
            ]
        \coordinate[label=below left:$O$] (O) at (0,0);
    
        % Plot 1
        \addplot [samples=1200, domain = 0:6.2831853] { cos(x) };
        \addplot [ dashed, samples=400, domain = -1:0] { cos(x) };
        \addplot [ dashed, samples=400, domain = 6.2831853:7] { cos(x) } node[below]{$y=\cos x$}; 

        %\draw [blue, dashed, latex-latex] (-1,0) --  node[left, blue]{amplitude} (-1,1) ;
        %\draw [blue, dashed, latex-latex] (-1,-1) --  node[left, blue]{amplitude} (-1,0) ;

        %\draw [purple, dashed, latex-latex] (0, 1.15) -- node[above, purple]{period} (6.283185307, 1.15);
        \end{axis}
    \end{tikzpicture}
    \end{center}

    The tangent ratio $\br{\tan}$ is defined as $ \dfrac{\text{sine}}{\text{cosine}}$ and is its graph is shown below.

    \begin{center}
    \begin{tikzpicture}[>=stealth]
        \begin{axis}[
            standard,
            xlabel = {$x$},
            ylabel = {$y$},
            xmin=-0.77, xmax=7.06,
            ymin=-4, ymax=4,
            xtick={1.570796327, 3.141592654, 4.71238898, 6.283185307}, 
            xticklabels={$90^{\circ}$, $180^{\circ}$, $270^{\circ}$, $360^{\circ}$}, 
            ytick={0},
            yticklabels={},
            height=8cm,
            width=12cm
            ]
        \coordinate[label=below left:$O$] (O) at (0,0);
    
        % Plot 1
        \addplot [dashed, samples=50, domain = -0.79:0] { tan(x) };
        \addplot [samples=125, domain = 0:1.32582] { tan(x) };
        \addplot [samples=250, domain = 1.81577:4.46741] { tan(x) };
        \addplot [samples=125, domain = 4.95737:2*pi] { tan(x) };
        \addplot [dashed, samples=50, domain = 2*pi:7.06] { tan(x) } node[above]{$y=\tan x$};

        \draw [dashed] (0.5*pi,-4.15) -- (0.5*pi,4.15);
        \draw [dashed] (1.5*pi,-4.15) -- (1.5*pi,4.15);
        \end{axis}
    \end{tikzpicture}
    \end{center}

\clearpage

We can summarise the behaviour of sine, cosine and tangent $\br{\dfrac{\text{sine}}{\text{cosine}}}$ ratios, in each quadrant below:

\begin{center}
    \begin{tikzpicture}[]
        \coordinate[label=below left:$O$] (O) at (0,0);
        \coordinate[] (P) at (1.6069, 1.9151);
        \coordinate[] (XR) at (2.5,0);
        
        \coordinate[label=above right:$1^{\text{st}}$ quadrant (A)] (A) at (1.7, 2);
        \coordinate[label=above left:$2^{\text{nd}}$ quadrant (S)] (B) at (-1.7, 2);
        \coordinate[label=below left:$3^{\text{rd}}$ quadrant (T)] (C) at (-1.7, -2);
        \coordinate[label=below right:$4^{\text{th}}$ quadrant (C)] (D) at (1.7, -2);

        \coordinate[label=below left:$180^{\circ}-\theta$ ] (B1) at (-2.5, 2);
        \coordinate[label=above left:$180^{\circ}+\theta$] (C1) at (-2.5, -2);
        \coordinate[label=above right:$360^{\circ}-\theta$] (D1) at (2.5, -2);
        
        \coordinate[label= above right: 
            $\begin{cases}
            \sin \theta >0 \\
            \cos \theta >0 \\
            \tan \theta >0 \\
        \end{cases}$
        ] (Q1) at (5,1.25);

        \coordinate[label= below right: 
            $\begin{cases}
            \sin \alpha <0 \\
            \textcolor{green!75!black}{\cos \alpha >0} \\
            \tan \alpha <0 \\
        \end{cases}$
        ] (Q4) at (5,-1.25);

\newenvironment{rcases}
  {\left.\begin{aligned}}
  {\end{aligned}\right\rbrace}

        \coordinate[label=above left: 
            $\begin{rcases}
            \textcolor{green!75!black}{\sin \alpha >0} \\
            \cos \alpha <0 \\
            \tan \alpha <0 \\
        \end{rcases}$
        ] (Q2) at (-5,1.25);

        \coordinate[label=below left: 
            $\begin{rcases}
            \sin \alpha <0 \\
            \cos \alpha <0 \\
            \textcolor{green!75!black}{\tan \alpha >0} \\
        \end{rcases}$
        ] (Q3) at (-5,-1.25);
        
        \draw [<-] (0,3) -- (0,-3);
        \draw [->] (-3,0) -- (3,0);
        \draw (0,0) circle (2.5cm);
        \fill (1.6069, 1.9151) circle (2pt);
        \draw (0,0) --  (P);
        \tkzMarkAngle[size=1, ->](XR,O,P); 
        \tkzLabelAngle[pos=0.7](XR,O,P){$\theta$};
        
    \end{tikzpicture}
\end{center}

You should also be familiar with the following exact values:

\begin{center}
    \renewcommand{\arraystretch}{1.75}
    \begin{tabular}{|C{1.5cm}|C{1.5cm}|C{1.5cm}|C{1.5cm}|C{1.5cm}|C{1.5cm}|}
    \hline
    $\theta$ & $0^{\circ}$ & $30^{\circ}$ & $45^{\circ}$ & $60^{\circ}$ & $90^{\circ}$  \\ 
    \hline
    $\sin$ & $0$ & $\dfrac{1}{2}$ & $\dfrac{1}{\sqrt{2}}$ & $\dfrac{\sqrt{3}}{2}$ & $1$  \\ 
    \hline
    $\cos$ & $1$ & $\dfrac{\sqrt{3}}{2}$ & $\dfrac{1}{\sqrt{2}}$ & $\dfrac{1}{2}$ & $0$  \\ 
    \hline
    $\tan$ & $0$ & $\dfrac{1}{\sqrt{3}}$ & $1$ & $\sqrt{3}$ & $\infty$  \\ 
    \hline

    \end{tabular}
    \end{center}


\end{theorybox}

\newpage

\begin{examplecz}{}
Find the exact value of:
\begin{enumerate}
    \begin{multicols}{3}
        \item $\sin 120^{\circ}$
        \item $\cos 210^{\circ}$
        \item $\tan 225^{\circ}$
        \item $\cos 315^{\circ}$
        \item $\tan 135^{\circ}$
        \item $\sin 330^{\circ}$
    \end{multicols}
\end{enumerate}


%\tcblower
%\textbf{Solution:}
%\vspace*{4.5cm}
%\begin{align*}
%    \dfrac{\sin \theta}{13} &= \dfrac{\sin 89 ^{\circ}}{25} \\
%    \sin \theta &= \dfrac{13 \sin 89^{\circ}}{25} \\
%    \therefore \theta &= \sin ^{-1} \left(\dfrac{13 \sin 89^{\circ}}{25}  \right) \\
%    &= 31^{\circ}
%\end{align*}
\end{examplecz}

\vspace{5mm}

\begin{examplecz}{}
Find the exact value of:
\begin{enumerate}
    \begin{multicols}{3}
        \item $\sin 90^{\circ}$
        \item $\cos 180^{\circ}$
        \item $\sin 270^{\circ}$
        \item $\cos 360^{\circ}$
        \item $\tan 180^{\circ}$
        \item $\tan 270^{\circ}$
    \end{multicols}
\end{enumerate}


\tcblower
\textbf{Solution:}
\vspace*{4.5cm}

%\begin{align*}
%    \dfrac{\sin \theta}{13} &= \dfrac{\sin 89 ^{\circ}}{25} \\
%    \sin \theta &= \dfrac{13 \sin 89^{\circ}}{25} \\
%    \therefore \theta &= \sin ^{-1} \left(\dfrac{13 \sin 89^{\circ}}{25}  \right) \\
%    &= 31^{\circ}
%\end{align*}
\end{examplecz}

\begin{comment}

\begin{examplecz}{}
Find the exact value of:
\begin{enumerate}
    \begin{multicols}{3}
        \item $\sin 570^{\circ}$
        \item $\cos 495^{\circ}$
        \item $\sin 690^{\circ}$
        \item $\cos 660^{\circ}$
        \item $\tan 405^{\circ}$
        \item $\sin \left(-270^{\circ} \right)$
    \end{multicols}
\end{enumerate}


\tcblower
\textbf{Solution:}
\vspace*{2cm}
%\begin{align*}
%    \dfrac{\sin \theta}{13} &= \dfrac{\sin 89 ^{\circ}}{25} \\
%    \sin \theta &= \dfrac{13 \sin 89^{\circ}}{25} \\
%    \therefore \theta &= \sin ^{-1} \left(\dfrac{13 \sin 89^{\circ}}{25}  \right) \\
%    &= 31^{\circ}
%\end{align*}
\end{examplecz}

\end{comment}

\newpage

\begin{examplecz}{}
Given that $\sin A = \dfrac{3}{7}$ and $\cos A <0$, find the exact value of $\tan A$.

\tcblower
\textbf{Solution:}
\vspace*{4.5cm}
%\begin{align*}
%    \dfrac{\sin \theta}{13} &= \dfrac{\sin 89 ^{\circ}}{25} \\
%    \sin \theta &= \dfrac{13 \sin 89^{\circ}}{25} \\
%    \therefore \theta &= \sin ^{-1} \left(\dfrac{13 \sin 89^{\circ}}{25}  \right) \\
%    &= 31^{\circ}
%\end{align*}
\end{examplecz}

\vspace{5mm}


\begin{examplecz}{}
Given that $\cos \theta = -\dfrac{5}{8}$ and $\tan \theta >0$, find the exact value of $\sin \theta$.

\tcblower
\textbf{Solution:}
\vspace*{4.5cm}
%\begin{align*}
%    \dfrac{\sin \theta}{13} &= \dfrac{\sin 89 ^{\circ}}{25} \\
%    \sin \theta &= \dfrac{13 \sin 89^{\circ}}{25} \\
%    \therefore \theta &= \sin ^{-1} \left(\dfrac{13 \sin 89^{\circ}}{25}  \right) \\
%    &= 31^{\circ}
%\end{align*}
\end{examplecz}

\vspace{5mm}

\begin{examplecz}{}
Given that $\tan \theta = \dfrac{4}{9}$ and $\sin \theta <0$, find the exact value of $\cos \theta$.

\tcblower
\textbf{Solution:}
\vspace*{4.5cm}
%\begin{align*}
%    \dfrac{\sin \theta}{13} &= \dfrac{\sin 89 ^{\circ}}{25} \\
%    \sin \theta &= \dfrac{13 \sin 89^{\circ}}{25} \\
%    \therefore \theta &= \sin ^{-1} \left(\dfrac{13 \sin 89^{\circ}}{25}  \right) \\
%    &= 31^{\circ}
%\end{align*}
\end{examplecz}

\begin{examplecz}{}
Give that $\sin \theta = 0.6$, find the value of:
\begin{enumerate}
    \begin{multicols}{3}
        \item $\sin \left( \theta - 180^{\circ} \right)$
        \item $\sin \left(\theta + 180^{\circ} \right)$
        \item $\sin \left( 360^{\circ} + \theta \right)$
        \item $\sin \left( 90^{\circ} + \theta \right)$
        \item $\sin \left( \theta - 90^{\circ} \right)$
        \item $\sin \left( 270^{\circ} - \theta \right)$
    \end{multicols}
\end{enumerate}


\tcblower
\textbf{Solution:}
\vspace*{4.5cm}
%\begin{align*}
%    \dfrac{\sin \theta}{13} &= \dfrac{\sin 89 ^{\circ}}{25} \\
%    \sin \theta &= \dfrac{13 \sin 89^{\circ}}{25} \\
%    \therefore \theta &= \sin ^{-1} \left(\dfrac{13 \sin 89^{\circ}}{25}  \right) \\
%    &= 31^{\circ}
%\end{align*}
\end{examplecz}

\vfill

\begin{Qbox}{}
    From \textit{CambridgeMaths Year 11 Mathematics Extension 1}:
    \begin{itemize}
        \item Exercise 6E: 1, 3, 4, 5, 6, 11
        \item Exercise 6F: 3, 7
    \end{itemize}
\end{Qbox}



\newpage

\subsection{Radian measure}


\begin{theorybox}{}

A radian is a unit of measure of angles. It is the constant angle that is formed at the centre of a circle when the arc length is equal to the radius.
\begin{center}
    \begin{tikzpicture}[scale=0.75]
        \draw[dashed] (0,0) circle (3cm);
        \draw[line width=1.2mm] (3,0) arc[start angle=0, end angle=57.2958,radius=3cm];
        \draw (0,0) -- node[below]{$r$} (3,0);
        \draw (0,0) -- node[above left]{$r$} (1.6209,2.5244);
        \coordinate[label=above right: $r$] (P) at (2.5981, 1.5);
        \coordinate[label=above right: $1^{c}$] (O) at (0.25,0);
    \end{tikzpicture}
\end{center}

Compare this sector to an equilateral triangle whose angles are all $60^{\circ}$.

\begin{center}
    \begin{tikzpicture}[scale=0.75]
        \draw[dashed, blue] (0,0) -- (5,0) -- (2.5, 4.3301) -- cycle;
        \draw[dotted, red,thick] (0,0) -- (5,0);
        \draw[dotted, red, thick] (2.7015, 4.2073) -- (0,0);
        \draw[thick, red, dotted] (5,0) arc[start angle=0, end angle=57.2958,radius=5cm];
        
    \end{tikzpicture}
\end{center}

We can see that $1^{c}$ is less than $60^{\circ}$. More accurately, $$1^{c} \approx 57.2958^{\circ}$$



Arranging three equilateral triangles, we can form one long straight edge.
\begin{center}
    \begin{tikzpicture}[scale=0.75]
        \draw[dashed, ] (0,0) -- (5,0) -- (2.5, 4.3301) -- cycle;
        \draw[dashed, ] (2.5, 4.3301)  -- (7.5, 4.3301) -- (5,0);
        \draw[dashed, ] (7.5, 4.3301) -- (10,0) -- (5,0) ;
        \draw [thick] (0,0) -- (10,0);
    \end{tikzpicture}
\end{center}

This is not the case when using the sectors with central angle $1^{c}$.
\begin{center}
    \begin{tikzpicture}[scale=0.75]
        \draw[dashed, ] (10,0) arc[start angle=0, end angle=57.2958,radius=5cm] -- (5,0);
        \draw[dashed, ] (7.70151,4.20735) arc[start angle=57.2958, end angle=114.5916,radius=5cm] -- (5,0);
        \draw[dashed, ] (2.9193,4.546487) arc[start angle=114.5916, end angle=171.8874,radius=5cm] -- (5,0);

        \draw [thick] (0,0) -- (10,0);
    \end{tikzpicture}
\end{center}

\clearpage

From the previous diagram, we can see that it takes a little bit more than $3^{\circ}$ to form $180^{\circ}$.
\vskip3mm
In fact, $180^{\circ} - 3 \times 57.2958^{\circ} \approx 8.1126^{\circ}$ remains.
\vskip3mm
$$\dfrac{8.1126}{57.2958} \approx 0.1415915 \dots$$
Hence, 
\begin{align*}
    3+0.1415915\dots \,  \text{radians} &=180^{\circ}   \\
    \therefore \pi \ \text{radians} &= 180^{\circ} 
\end{align*}

In practice, we do not write \textit{`radians'} so we often see $$\pi \cancel{\text{ radians}} = 180^{\circ}$$

The diagram below summarises the relationship between degrees and radians

\begin{center}
\begin{tikzpicture}
    \node [] (A)              {Degrees};
    \node [] (B) [right=of A] {Radians};
    
    \draw [thick, ->]
    (A) edge [bend left=45] node[above]{$\times \dfrac{\pi}{180}$} (B)
    (B) edge [bend left=45] node[below]{$\times \dfrac{180}{\pi}$} (A);
    
\end{tikzpicture}
\end{center}


\end{theorybox}

\vspace{5mm}

\begin{examplecz}[height fill=true]{}
    Express the following angles in radians, leaving your answer in terms of $\pi$.
    \begin{enumerate}
    \begin{multicols}{3}
        \item $360^{\circ}$
        \item $270^{\circ}$
        %\item $180^{\circ}$
        \item $120^{\circ}$
        \item $90^{\circ}$
        \item $45^{\circ}$
        %\item $60^{\circ}$
        %\item $30^{\circ}$
        \item $540^{\circ}$
    \end{multicols}
    \end{enumerate}
\tcblower
\textbf{Solution:}
%\vspace*{cm}
\end{examplecz}

\begin{examplecz}{}
    Express the following angles in radians, correct to three decimal places.
    \begin{enumerate}
    \begin{multicols}{3}
        \item $360^{\circ}$
        %\item $270^{\circ}$
        %\item $180^{\circ}$
        %\item $120^{\circ}$
        %\item $150^{\circ}$
        %\item $400^{\circ}$
        %\item $210^{\circ}$
        %\item $105^{\circ}$
        %\item $315^{\circ}$
        \item $17^{\circ}$
        \item $62.8^{\circ}$
    \end{multicols}
    \end{enumerate}
\tcblower
\textbf{Solution:}
\vspace{60mm}
\end{examplecz}

\vspace{5mm}

\begin{examplecz}{}
    Express the following angles in degrees, to the nearest minute.
    \begin{enumerate}
    \begin{multicols}{3}
        \item $\dfrac{5 \pi}{4}$
        \item $0.6$
        \item $\dfrac{4 \pi}{5}$
        \item $\dfrac{3 \pi}{2}$
        \item $4.35$
        \item $\dfrac{ \pi}{6}$
        %\item $\dfrac{ \pi}{3}$
        %\item $\dfrac{3}{10}$
        %\item $1.6$
        
    \end{multicols}
    \end{enumerate}
\tcblower
\textbf{Solution:}
\vspace{60mm}
\end{examplecz}

\vfill

\begin{Qbox}{}
    From \textit{CambridgeMaths Year 11 Mathematics Extension 1}:
    \begin{itemize}
        \item Exercise 11G: 1 to 8, 11
    \end{itemize}
\end{Qbox}

\newpage

\subsection{Arcs, sectors and segments}

\begin{theorybox}{}

Using radians, we can deduce other formulae to calculate certain properties of circles and sectors.
\vskip5mm
The arc length, $l$ units, subtended by an angle $\theta$ (radians) in a circle of radius $r$ units is given by $$l=r\theta$$
\begin{center}
    \begin{tikzpicture}[scale=0.75]
        \draw[] (0,0) circle (3cm);
        \fill (0,0) circle (1.5pt);
        \draw[line width=1.2mm] (3,0) arc[start angle=0, end angle=57.2958,radius=3cm];
        \draw [dashed] (0,0) -- node[below]{$r$} (3,0);
        \draw[dashed] (0,0) -- node[above left]{$r$} (1.6209,2.5244);
        \coordinate[label=above right: $l$] (P) at (2.5981, 1.5);
        \coordinate[label=above right: $\theta$] (O) at (0.25,0);
    \end{tikzpicture}
\end{center}

\vskip5mm
The area, $A\ \text{units}^{2}$, of a sector subtended by an angle $\theta$ (radians) in a circle of radius $r$ units is given by $$A=\dfrac{1}{2} r^2 \theta$$
\begin{center}
    \begin{tikzpicture}[scale=0.75]
        \fill (0,0) circle (1.5pt);
        \fill[color=gray!20!white] (0,0) -- (3,0) arc[start angle=0, end angle=57.2958,radius=3cm] -- (0,0);
        \draw[] (0,0) circle (3cm);
        \draw (0,0) -- node[below]{$r$} (3,0);
        \draw (0,0) -- node[above left]{$r$} (1.6209,2.5244);
        \coordinate[label=above right: $\theta$] (O) at (0.25,0);
    \end{tikzpicture}
\end{center}

\vskip5mm
The area, $A\ \text{units}^{2}$, of a minor segment subtended by an angle $\theta$ (radians) in a circle of radius $r$ units is given by 
$$A=\dfrac{1}{2} r^2 \left( \theta - \sin \theta \right)$$
\begin{center}
    \begin{tikzpicture}[scale=0.75]
        \fill (0,0) circle (1.5pt);
        \fill[color=gray!20!white]  (3,0) arc[start angle=0, end angle=57.2958,radius=3cm] -- (1.6209,2.5244) -- cycle;
        \draw[] (0,0) circle (3cm);
        \draw[dashed] (3,0) -- node[below]{$r$} (0,0) -- node[above left]{$r$} (1.6209,2.5244);
        \draw[] (3,0) -- (1.6209,2.5244);
        \coordinate[label=above right: $\theta$] (O) at (0.25,0);
    \end{tikzpicture}
\end{center}

\end{theorybox}

\newpage

\begin{examplecz}[height fill = true]{}
    A circle centred at $O$ has a radius of $6$\,cm. Points $A$ and $B$ lie on the circumference of the circle such that $\angle AOB = 135^{\circ}$.
    \begin{center}
    \begin{tikzpicture}[]
        \draw[] (0,0) circle (3cm);
        \draw [dashed] (0,0) -- node[below]{$6$\,cm} (3,0) node[right]{$A$};
        \draw[dashed] (0,0) -- (-2.1213, 2.1213) node[above left]{$B$};
        \coordinate[label=below left: $O$] (O) at (0,0);
        \coordinate[] (A) at (3,0);
        \coordinate[] (B) at (-2.1213, 2.1213);
        \tkzLabelAngle[pos=0.7](A,O,B){$135^{\circ}$};
        \fill (0,0) circle (2pt);
    \end{tikzpicture}
\end{center}
    \begin{enumerate}
        \item Find the length of the minor arc $AB$, correct to one decimal place.
        \item Find the area of the sector $OAB$, correct to two decimal places.
    \end{enumerate}
\tcblower
\textbf{Solution:}

\end{examplecz}

\newpage
\begin{examplecz}{}
    The radius of a circle is $3$\,cm and an arc is $0.27\pi$\,cm long. 
    \vskip3mm
    Find the angle subtended at the centre of the circle by the arc, correct to two decimal places.
\tcblower
\textbf{Solution:}
\vspace*{60mm}
    
\end{examplecz}

\vspace*{5mm}
%\begin{examplecz}{}
%    The circumference of a circle is $300$\,mm.
%    \vskip3mm
%    Find the length of the arc that is formed by an angle of $\dfrac{\pi}{6}$ subtended at the centre of the circle.
%\end{examplecz}

%\vspace*{5mm}
%\begin{examplecz}{}
%    Find the angle that is subtended at the centre of the circle by an arc that is $8$\,cm long.
%\end{examplecz}


\begin{examplecz}{}
    The area of the sector of a circle that is subtended by an angle of
    $\dfrac{\pi}{3}$ at the centre is $6\pi\,\text{cm}^2$. 
    \vskip3mm
    Find the radius of the circle.
\tcblower
\textbf{Solution:}
\vspace*{60mm}
\end{examplecz}

%\vspace*{5mm}
%\begin{examplecz}{}
%    A sector of a circle with radius $5$\,cm and an angle of $\dfrac{\pi}{3}$ subtended at the centre is cut out of cardboard. It is then curved around to form a cone.
%    \vskip3mm
%    Find its exact surface area and volume.
%\end{examplecz}

\newpage
\begin{examplecz}{}
    The area of a sector is $\dfrac{3\pi}{10}\,\text{cm}^2$ and the arc length cut off by the sector is $\dfrac{\pi}{5}\,\text{cm}$
    \vskip3mm
    Find the angle subtended at the centre, and the radius, of the circle.
\tcblower
\textbf{Solution:}
\vspace*{60mm}
\end{examplecz}

\vspace*{5mm}
\begin{examplecz}{}
    Find the area of the minor segment formed by an angle of $40^{\circ}$ subtended at the centre of a circle with radius $2.82$\,cm, correct to two significant figures.
\tcblower
\textbf{Solution:}
\vspace*{60mm}
\end{examplecz}

\newpage

\begin{examplecz}{}
    An angle of $\dfrac{\pi}{8}$ is subtended at the centre of a circle. This angle cuts off an arc of $54\pi$\,cm.
    \begin{enumerate}
        \item Find the exact area of the sector.
        \item Find the area of the minor segment formed, correct to 1 decimal place.
    \end{enumerate}
\tcblower
\textbf{Solution:}
\vspace*{80mm}
\end{examplecz}

\newpage


\begin{examplecz}[]{}
    A circular chocolate wafer of radius $2$\,cm is dipped in chocolate so that the length of the straight edge measures $3$\,cm as shown.
    \begin{center}
        \begin{tikzpicture}[scale=1]
            
            \fill[color=brown!75!white]  (-2.4575,-1.720) arc[start angle=215, end angle=325,radius=3cm] -- (2.4575,-1.720) -- cycle;
            \draw[] (0,0) circle (3cm);
            
            \draw[dashed] (-2.4575,-1.720) -- node[above left]{$2$\,cm} (0,0) -- node[above right]{$2$\,cm} (2.4575,-1.720);
            
            \draw[] (-2.4575,-1.720) -- node[below]{$3$\,cm} (2.4575,-1.720);
            \fill (0,0) circle (2pt);
            
        \end{tikzpicture}
    \end{center}

    Find the area of the wafer that is not covered in chocolate, correct to two decimal places.
    \tcblower
\textbf{Solution:}
\vspace*{100mm}
\end{examplecz}

\begin{comment}
\begin{examplecz}{}
    Point $A$ on a circular wheel of radius $1$ metre sits on a horizontal surface. Point $B$ is lies at an arc length of $2$ metres from $A$ as shown.
    \begin{center}
    \begin{tikzpicture}[]
        \draw[] (0,0) circle (3cm);
        \draw (-4,-3) -- (4,-3);
        \draw[dashed, ->] (-2.3,2.3) arc (135:270:3.25) node[midway,left]{$2$\,m}; angle=270,radius=3.25cm];
        \draw [dashed] (0,0) -- node[right]{$1$\,m} (0,-3) node[below right]{$A$};
        \draw[dashed] (0,0) -- (-2.1213, 2.1213) node[above]{$B$};
        \fill (0,0) circle (2pt) node[above right]{$O$};
        \coordinate[] (A) at (3,0);
        \coordinate[] (B) at (-2.1213, 2.1213);
    \end{tikzpicture}
\end{center}

    \begin{enumerate}
        \item Find the size of $\angle AOB$, correct to three decimal places.
        \item The wheel turns counter-clockwise, without slipping, so that point $B$ rests on the surface.
        \vskip3mm
        Find the vertical height of point $A$ above the ground, correct to one decimal place.
    \end{enumerate}
\end{examplecz}
\end{comment}

\vfill

\begin{Qbox}{}
    From \textit{CambridgeMaths Year 11 Mathematics Extension 1}:
    \begin{itemize}
        \item Exercise 11I: 1 to 16
    \end{itemize}
\end{Qbox}

\newpage
\subsection{Trigonometric equations}

\begin{theorybox}{}
Solving trigonometric equations requires two main steps:
\begin{itemize}
    \item \textbf{Step 1:} Find the \textit{reference angle} in the first quadrant.
    \item \textbf{Step 2:} Find the \textit{related angles} in remaining quadrants.
\end{itemize}
   
\end{theorybox}

\vspace{5mm}
\begin{examplecz}[height fill = true]{}

Solve the following equations, where $ 0^{\circ} \leq \theta \leq 360^{\circ}$.

\begin{enumerate}
\begin{multicols}{3}
    \item $ \sin \theta = \dfrac{\sqrt{3}}{2}$
    \item $ \cos \theta = \dfrac{\sqrt{3}}{2}$
    \item $ \tan \theta = -\dfrac{1}{\sqrt{3}}$
\end{multicols}
\end{enumerate}
\tcblower
\textbf{Solution:}

\end{examplecz}

\newpage
\begin{examplecz}[height fill = true]{}

Solve the following equations, where $ 0^{\circ} \leq \theta \leq 360^{\circ}$.

\begin{enumerate}
\begin{multicols}{3}
    \item $ 2\sin \theta = -1$
    \item $ 2\cos \theta = \sqrt{3}$
    \item $ \tan \theta - \sqrt{3} = 0 $
\end{multicols}
\end{enumerate}
\tcblower
\textbf{Solution:}

\end{examplecz}

\newpage
\begin{examplecz}[]{}

Solve the following equations, where $ 0^{\circ} \leq \theta \leq 360^{\circ}$.

\begin{enumerate}
\begin{multicols}{3}
    \item $ \sin \theta =0$
    \item $ 4\sin ^2 \theta  - 3= 0$
    \item $ \tan \theta = \dfrac{4}{9}$
\end{multicols}
\end{enumerate}
\tcblower
\textbf{Solution:}
\vspace{15cm}

    
\end{examplecz}

\vfill

\begin{Qbox}{}
    From \textit{CambridgeMaths Year 11 Mathematics Extension 1}:
    \begin{itemize}
        \item Exercise 11I: 1 to 4, 7 and 8
    \end{itemize}
\end{Qbox}



\newpage
\subsection*{Graphs of the Reciprocal Ratios}

\begin{theorybox}{}
The diagrams below show the graphs of $y= \text{cosec}\,{x}$, $y=\sec x$ and $ y = \cot x$ for $ 0 \leq x \leq 2\pi$.

\begin{center}
    \begin{tikzpicture}[>=stealth]
        \begin{axis}[
            clip=false,
            standard,
            xlabel = {$x$},
            ylabel = {$y$},
            xmin=0, xmax=2*pi,
            ymin=-3, ymax=3,
            xtick={pi/2, 3*pi/2}, 
            xticklabels={$\frac{\pi}{2}$, $\frac{3\pi}{2}$}, 
            ytick={-1,1},
            yticklabels={$-1$, $1$},
            height=8cm,
            width=12cm
            ]
        \coordinate[label=below left:$O$] (O) at (0,0);
    
        % Plot 1
        \addplot [samples=750, domain = 0.34:2.80] { cosec(x) } node[ above right]{$y=\text{cosec}\,x$};
       \addplot [samples=750, domain = 3.48:5.94] { cosec(x) };
        \draw [dashed,] (pi,-3) -- (pi,0)  node[below left]{$\pi$} -- (pi,3) ;
        \draw [dashed,] (2*pi,-3) -- (2*pi,0)  node[below left]{$2\pi$} -- (2*pi,3) ;
        \end{axis}
    \end{tikzpicture}
    \end{center}

\begin{center}
    \begin{tikzpicture}[>=stealth]
        \begin{axis}[
            clip=false,
            standard,
            xlabel = {$x$},
            ylabel = {$y$},
            xmin=0, xmax=2*pi,
            ymin=-3, ymax=3,
            xtick={pi, 2*pi}, 
            xticklabels={$\pi$, $2\pi$}, 
            ytick={-1,1},
            yticklabels={$-1$, $1$},
            height=8cm,
            width=12cm
            ]
        \coordinate[label=below left:$O$] (O) at (0,0);
    
        % Plot 1
        \addplot [samples=500, domain = 0:1.23] { sec(x) } node[ above right]{$y=\text{sec}\,x$};
        \addplot [samples=500, domain = 1.91:4.37] { sec(x) };
        \addplot [samples=500, domain = 5.05:2*pi] { sec(x) };
        \draw [dashed,] (pi/2,-3) -- (pi/2,0)  node[below left]{$\frac{\pi}{2}$} -- (pi/2,3) ;
        \draw [dashed,] (3*pi/2,-3) -- (3*pi/2,0)  node[below left]{$\frac{3\pi}{2}$} -- (3*pi/2,3) ;
        \end{axis}
    \end{tikzpicture}
    \end{center}

    \begin{center}
    \begin{tikzpicture}[>=stealth]
        \begin{axis}[
            standard,
            xlabel = {$x$},
            ylabel = {$y$},
            xmin=0, xmax=2*pi,
            ymin=-4, ymax=4,
            xtick={pi/2, 3*pi/2}, 
            xticklabels={ $\frac{\pi}{2}$, $\frac{3\pi}{2}$}, 
            ytick={0},
            yticklabels={},
            height=8cm,
            width=12cm
            ]
        \coordinate[label=below left:$O$] (O) at (0,0);
    
        % Plot 1
        \addplot [samples=250, domain = 0.24:2.897] { cot(x) };
        \addplot [samples=250, domain = 3.386:6.038] { cot(x) } node[below left]{$y=\cot x$};


        \draw [dashed] (pi,-4) -- (pi,0) node[below left]{$\frac{\pi}{2}$}-- (pi,4);
        \draw [dashed] (2*pi,-4) -- (2*pi,0) node[below left]{$2\pi$} --(2*pi,4);
        \end{axis}
    \end{tikzpicture}
    \end{center}

\clearpage

We can summarise the behaviour of sine, cosine and tangent $\br{\dfrac{\text{sine}}{\text{cosine}}}$ ratios, in each quadrant below:

\begin{center}
    \begin{tikzpicture}[]
        \coordinate[label=below left:$O$] (O) at (0,0);
        \coordinate[] (P) at (1.6069, 1.9151);
        \coordinate[] (XR) at (2.5,0);
        
        \coordinate[label=above right:$1^{\text{st}}$ quadrant (A)] (A) at (1.7, 2);
        \coordinate[label=above left:$2^{\text{nd}}$ quadrant (S)] (B) at (-1.7, 2);
        \coordinate[label=below left:$3^{\text{rd}}$ quadrant (T)] (C) at (-1.7, -2);
        \coordinate[label=below right:$4^{\text{th}}$ quadrant (C)] (D) at (1.7, -2);

        \coordinate[label=below left:$180^{\circ}-\theta$ ] (B1) at (-2.5, 2);
        \coordinate[label=above left:$180^{\circ}+\theta$] (C1) at (-2.5, -2);
        \coordinate[label=above right:$360^{\circ}-\theta$] (D1) at (2.5, -2);
        
        \coordinate[label= above right: 
            $\begin{cases}
            \sin \theta >0 \\
            \cos \theta >0 \\
            \tan \theta >0 \\
        \end{cases}$
        ] (Q1) at (5,1.25);

        \coordinate[label= below right: 
            $\begin{cases}
            \sin \alpha <0 \\
            \textcolor{green!75!black}{\cos \alpha >0} \\
            \tan \alpha <0 \\
        \end{cases}$
        ] (Q4) at (5,-1.25);

\newenvironment{rcases}
  {\left.\begin{aligned}}
  {\end{aligned}\right\rbrace}

        \coordinate[label=above left: 
            $\begin{rcases}
            \textcolor{green!75!black}{\sin \alpha >0} \\
            \cos \alpha <0 \\
            \tan \alpha <0 \\
        \end{rcases}$
        ] (Q2) at (-5,1.25);

        \coordinate[label=below left: 
            $\begin{rcases}
            \sin \alpha <0 \\
            \cos \alpha <0 \\
            \textcolor{green!75!black}{\tan \alpha >0} \\
        \end{rcases}$
        ] (Q3) at (-5,-1.25);
        
        \draw [<-] (0,3) -- (0,-3);
        \draw [->] (-3,0) -- (3,0);
        \draw (0,0) circle (2.5cm);
        \fill (1.6069, 1.9151) circle (2pt);
        \draw (0,0) --  (P);
        \tkzMarkAngle[size=1, ->](XR,O,P); 
        \tkzLabelAngle[pos=0.7](XR,O,P){$\theta$};
        
    \end{tikzpicture}
\end{center}

You should also be familiar with the following exact values:

\begin{center}
    \renewcommand{\arraystretch}{1.75}
    \begin{tabular}{|C{1.5cm}|C{1.5cm}|C{1.5cm}|C{1.5cm}|C{1.5cm}|C{1.5cm}|}
    \hline
    $\theta$ & $0^{\circ}$ & $30^{\circ}$ & $45^{\circ}$ & $60^{\circ}$ & $90^{\circ}$  \\ 
    \hline
    $\sin$ & $0$ & $\dfrac{1}{2}$ & $\dfrac{1}{\sqrt{2}}$ & $\dfrac{\sqrt{3}}{2}$ & $1$  \\ 
    \hline
    $\cos$ & $1$ & $\dfrac{\sqrt{3}}{2}$ & $\dfrac{1}{\sqrt{2}}$ & $\dfrac{1}{2}$ & $0$  \\ 
    \hline
    $\tan$ & $0$ & $\dfrac{1}{\sqrt{3}}$ & $1$ & $\sqrt{3}$ & $\infty$  \\ 
    \hline

    \end{tabular}
    \end{center}


\end{theorybox}


    \newpage
\section{Trigonometric Identities}

\subsection{Pythagorean Identities}

\begin{theorybox}{}
The reciprocal ratios are:
\begin{itemize}
    \begin{multicols}{3}
        \item $\text{cosec} \theta = \dfrac{1}{\sin \theta}$
        \item $ \sec \theta = \dfrac{1}{\cos \theta} $
        \item $ \cot \theta = \dfrac{\cos \theta}{\sin \theta} $
    \end{multicols}
\end{itemize}
The first Pythagorean theorem states that
\begin{align}
    \sin ^2 \theta + \cos ^2 \theta =1
\end{align}
We can arrange this to state
\begin{align*}
    \sin ^2 \theta &= 1 - \cos ^2 \theta \qquad \text{or} \\
    \cos ^2 \theta &= 1 - \sin ^2 \theta 
\end{align*}

Using $(1)$, we can also obtain two other trigonometric identities
\begin{align}
    1 + \cot ^2 \theta =\text{cosec}^2\theta
\end{align}

and
\begin{align}
    1  + \tan ^2 \theta =\sec^2\theta
\end{align}


\end{theorybox}

\begin{pfbox}{}
    Consider the circle $x^2 +y^2 =r^2$ in the diagram below.
    \begin{center}{}
        \begin{tikzpicture}[>=latex, scale=0.8]
            % x-axis
            \draw[->](-4,0) -- (4,0) node[below ]{$x$};
            % y-axis
            \draw[->](0,-4) -- (0,4) node[left]{$y$};
            % Circle
            \draw (0,0) circle (3cm);
            \coordinate[label=below right:$r$] (A) at (3,0);
            \coordinate[label=below left:$-r$] (B) at (-3,0);
            \coordinate[label=above left:$r$] (C) at (0,3);
            \coordinate[label=below left:$-r$] (D) at (0,-3);
            \coordinate[label=below left:$O$] (O) at (0,0);
            \draw[] (0,0) -- (1.928, 2.298) node[above right]{$ \left( r \cos \theta ,\, r \sin \theta \right)$};
            \fill (1.928, 2.298) circle (2pt);        
        \end{tikzpicture}
    \end{center}

    The coordinates of the point must satisfy the equation of the circle, hence
    \begin{align*}
        \left(r \cos \theta  \right)^2 + \left(r \sin \theta \right)^2 &= r^2 \\
        r^2 \cos^2 \theta + r^2 \sin ^2 \theta &= r^2 \\
        \therefore \cos ^2 \theta + \sin ^2 \theta &= 1
    \end{align*}
    To obtain (2), we divide (1) by $\sin^2 \theta$ and to obtain (3), we divide (1) by $\cos^2 \theta$.
    
\end{pfbox}

\newpage

\begin{examplecz}{}
Prove that: 
\vskip5mm
    \begin{enumerate}
    \setlength{\itemsep}{8mm}

        %\item $\cot^2 \theta = \text{cosec}^2 \theta -1 $
        \item $3 + 3 \tan^2 \alpha = \dfrac{3}{1- \sin^2 \alpha} $
        \item $ \sec ^2 x - \tan ^2 x = \text{cosec}^2 x - \cot^2 x $

        \item $ \cot A + 2 \sec A = \dfrac{1-\sin ^2 A +2 \sin A}{\sin A \cos A}  $
        \item $ \dfrac{1 - \sin^2 A \cos ^2 A }{\cos ^2 A} = \tan ^2 A + \cos^2 A$
        \item $\sec^2 A \left(1 - \sin ^2 A \right) = 1$
        \item $ \dfrac{1}{1 + \sin ^2 \theta } + \dfrac{1}{1+ \text{cosec}^2\theta } = 1  $
        \item $ \dfrac{\sin A - \cos A}{\text{cosec}A - \sin A} = \tan ^2 A - \tan A  $
    \end{enumerate}
\end{examplecz}

\vskip10mm
\begin{examplecz}{}
Simplify:
\vskip5mm
    \begin{enumerate}
    \setlength{\itemsep}{8mm}
    \begin{multicols}{2}
        \item $ \sec \theta \cot \theta $
        \item $5 \cot^2 A +5 $
        \item $ \cot x - \cot x \cos^2 x $
        \item $ \sin^2 A \text{cosec}^2 A $
    \end{multicols}
    \end{enumerate}

\end{examplecz}

\begin{examplecz}{}
Find the value(s) of $x$, for $0 \leq x \leq 2\pi$, such that:
\vskip5mm
    \begin{enumerate}
    \setlength{\itemsep}{8mm}
        \item $ \sin^2 x = \cos^2 x -1$
        \item $ 7 \cos x -2 \sin^2 x = 2 $
        \item $ \dfrac{3}{2} \cos^2 x + \sin x =1 $
    \end{enumerate}

\end{examplecz}





\newpage
\subsection{Identity Marathon}


\begin{questions}

\begin{Qbox}

\begin{questions}

\uplevel{Prove that:}

\question $\cos 60 ^{\circ} + \cos 120 ^{\circ} - \cos 180 ^{\circ} = 1-4 \sin 30 ^{\circ} \sin 60 ^{\circ} \cos 90 ^{\circ}$

\question $\tan 60 ^{\circ} + \tan 120 ^{\circ} + \tan 180 ^{\circ} = \cot 210 ^{\circ} \cot 330 ^{\circ} \tan 180 ^{\circ}$

\question $\left( \sin 150 ^{\circ} + \cos 270 ^{\circ} + \tan 315 ^{\circ} \right)^{2} = \sin ^{2} 135 ^{\circ} \cos ^{2} 225 ^{\circ}$

\question $\sin ^{4} 120 ^{\circ} + \cos ^{4} 120 ^{\circ} = 1 - 2 \sin ^{2} 120 ^{\circ} \cos ^{2} 120 ^{\circ}$

\question $\cos ^{2} 45 ^{\circ} \cos ^{2} 150 ^{\circ} - \sin ^{2} 45 ^{\circ} \sin ^{2} 150 ^{\circ} = \left(\cos ^{2} 45 - \sin ^{2} 120 ^{\circ} \right) \tan 135 ^{\circ}$

\question $\dfrac{\sin 120 ^{\circ}}{ \tan 300 ^{\circ}} - \dfrac{\cos 240 ^{\circ}}{\cot 315 ^{\circ}} = \tan^{2} 240 ^{\circ} - \text{cosec} ^{2} 330 ^{\circ}$

\vskip10mm
\uplevel{Simplify:}
\vskip5mm
\begin{multicols}{2}

\question $\displaystyle \frac{\sin^{2} \theta + \cos^{2} \theta}{\tan ^{2} \theta}$

\question $\displaystyle \frac{2 \cot \alpha}{1 + \cot^{2} \alpha}$

\question $\displaystyle \frac{\sin A}{\cos A} + \frac{\cos A}{\sin A}$

\question $\text{cosec}^{2} \theta - \cot^{2} \theta$

\question $\displaystyle \frac{\sin ^{2} \theta}{1 - \sin^{2} \theta}$

\question $\big(\sec^{2} \theta -1 \big)\tan \big(90 ^{\circ} - \theta \big)$

\question $\sin ^{3} \theta + \sin \theta \cos ^{2} \theta$

\question $1-\sin ^{2} \big(180 + \theta \big) $

\question $\big(1 + \tan^{2}u \big) \big(1 - \sin^{2} u \big)$

\question $\displaystyle \frac{\sin \theta}{1 + \cos \theta } + \frac{1 + \cos \theta}{\sin \theta}$

\question $\sin \theta \cos \big(90 ^{\circ} - \theta \big) + \cos \theta \sin \big( 90 ^{\circ} - \theta \big)$

\question $2 \cos ^{2} 30 ^{\circ} -1 $

\question $\displaystyle \frac{1}{1 - \sin \gamma} + \frac{1}{1+\sin \gamma}$

\question $\displaystyle \frac{1 - \sin \theta }{1 + \cos \theta} \times \frac{1+ \sin \theta}{1 - \cos \theta}$

\question $\displaystyle 1 - \frac{\sin A \cos A}{\tan A}$

\question $1- \sin \theta \cos \big( 90 ^{\circ} - \theta \big)$

\end{multicols}


\newpage
\uplevel{Prove that:}

\question $\tan \big( 90 ^{\circ} - A \big) \sec \big( 180 ^{\circ} + A \big) \cos \big( 90 ^{\circ} + A \big) = 1$

\question $\sin \big(360 ^{\circ} -A \big) \text{cosec} \big(180^{\circ} + A \big) \cot \big(270 ^{\circ} - A \big) = \cot \big(90 ^{\circ} - A \big)$

\question $\tan \big(180 ^{\circ} - A \big) \sin \big( 270 ^{\circ} + A \big) \text{cosec} \big( 360 ^{\circ} -A \big) = -1 $

\question $\sin \big( 360 ^{\circ} - A \big) + \tan \big( 270 ^{\circ} + A \big) - \sin \big( 180 ^{\circ} + A \big) = \tan \big( 90 ^{\circ} + A \big)$

\question $\cos \big( 180 ^{\circ} - A \big) - \sin \big( 270 ^{\circ} -A \big) + \cos \big( 90 ^{\circ} - A \big) - \cos \big(270 ^{\circ} + A \big) =0$

\question $\text{cosec} \big(180 ^{\circ} + A \big) + \sec \big( 90 ^{\circ} - A \big) - \cot \big( 360 ^{\circ} - A \big) = \tan \big( 270 ^{\circ} - A \big)$

\vskip10mm
\uplevel{Prove the following identities:}
\vskip 5mm
\begin{multicols}{2}

\question $\big(1-\tan x \big)^{2} + \big( 1+ \tan x\big) ^{2} = 2 \sec ^{2}x$

\question $\displaystyle \big( \cot x + \text{cosec} x \big)^{2} = \frac{1 + \cos x }{1- \cos x}$

\question $\sin^{2} \alpha \cos^{2} \beta - \cos^{2} \alpha \sin ^{2} \beta \\ = \sin^{2} \alpha - \sin^{2} \beta$

\question $\displaystyle \sec \theta + \tan \theta = \frac{1 + \sin \theta}{\cos \theta}$

\question $\displaystyle \sin A \big( 1 + \tan A \big) + \cos A \big(1 + \cot A \big) \\ = \frac{\sin A + \cos A}{\sin A\cos A}$

\question $\displaystyle \frac{\cot \theta \cos \theta }{\cot \theta + \cos \theta} = \frac{\cos \theta}{1 + \sin \theta}$

\question $\displaystyle \frac{1 + \cot \theta}{\text{cosec} \theta } - \frac{\sec \theta}{\tan \theta + \cot \theta} = \cos \theta$


\question $\sin A = \cos A \tan A$

\question $\cos A = \sin A \cot A$

\question $\tan A = \sin A \sec A$

\question $\cot \phi = \cos \phi \, \text{cosec} \phi$

\question $ \big( 1 + \tan ^{2} \alpha \big) \cos ^{2} \alpha =1 $

\question $\text{cosec} ^{2} \beta \sin \beta \cos \beta = \cot \beta$

\question $\cot^{2} \beta \sec \beta = \cos \beta \, \text{cosec} ^{2} \beta$

\question $\big( \cos A + \cot A \big) \sec A = 1 + \text{cosec} A$

\question $\cot \theta \big( \sec ^{2} -1 \big) = \tan \theta $

\question $\displaystyle \frac{\sec \alpha - 1 }{\sec \alpha + 1 }= \frac{1 - \cos \alpha}{1 + \cos \alpha}$

\question $\tan A \big(1- \cot^{2}A \big)+ \cot A \big( 1 - \tan^{2} A \big) =0$

\question $\displaystyle \frac{\big( \cos t \cot t - \sin t \tan t\big) \sin t \cos t}{\cos t - \sin t}\\ = 1 + \sin t \cos t$

\question $\sin ^{2} A \tan A + \cos^{2} A \cot A + 2 \sin A \cos A \\ = \tan A + \cot A$

\end{multicols}

\newpage
\uplevel{Prove the following identities:}
\vskip 5mm
\begin{multicols}{2}

\question $\big( \sin \phi + \cos \phi \big)^{2} = 1 + 2 \sin \phi \cos \phi$

\question $\big(\cos \mu - \sin \mu \big)^{2} = 1- 2 \sin \mu \cos \mu$

\question $\dfrac{\tan A}{1 - \tan^{2} A} = \dfrac{\sin A \cos A}{\cos ^{2}A - \sin^{2}A}$

\question $\dfrac{\tan A}{1 + \tan ^{2} A} = \sin A \cos A$

\question $\dfrac{\sin \theta}{\cot \theta \sec \theta} = 1 - \cos ^{2}$

\question $\dfrac{\big( 1 - \sin A \big) \big(1 + \sin A \big)}{\sin^{2}A} = \cot ^{2} A$ 


\question $\cot \alpha + \tan \alpha = \sec \alpha\,
\text{cosec} \alpha$

\question $\dfrac{\cot A + \cot B}{\tan A + \tan B} = \cot A\, \cot B$

\question $\sin B + \cot B \cos B = \text{cosec} B$

\question $\dfrac{\cot \theta - \tan \theta}{\cot \theta + \tan \theta} = \cos^{2} \theta - \sin^{2} \theta$

\question $\dfrac{1}{1 + \sin \phi} + \dfrac{1}{1-\sin \phi} = 2 \sec ^{2} \phi$

\question $\dfrac{1}{\sec A + \tan A} = \sec A - \tan A$

\question $\dfrac{\cos A - \tan A \sin A}{\cos A + \tan A \sin A} = 1 - 2 \sin ^{2} A$

\question $ \big( \tan A + \cot A \big)^{2} + \big(\sin A + \text{cosec} A \big) ^{2} + \big( \cos A + \sec A \big)^{2} = 5+ 2 \sec^{2} A + 2 \text{cosec} ^{2} A$

\question $ \dfrac{\sin A \cos B + \cos A \sin B}{\cos A \cos B - \sin A \sin B} = \dfrac{\tan A + \tan B}{1 - \tan A \tan B}$

\end{multicols}

\end{questions}

\vspace*{10mm}

\end{Qbox}
    
\end{questions}

\nomorequestions

    \newgeometry{right=30mm, left=15mm, top=20mm, bottom=20mm,}
\lhead{Name:}
\section*{Topic Test:}
\subsection*{MA-T1 Trigonometry and Measure of Angles}
\subsection*{MA-T2 Trigonometric Functions and Identities}

\vpword{Marks}
%\multicolumngradetable{3}[questions]

\begin{questions}
\question[1]
    Find the value of $\theta$ such that $\cos ( \theta + 25^{\circ}) = \sin (10^{\circ} + \theta) $.
    \fillwithdottedlines{25mm}

    \question Find the exact value of:
    \begin{parts}
        \part[1] $\tan \dfrac{7 \pi}{4}$.
        \fillwithdottedlines{17mm}
        \part[1] $\cos 930^{\circ}$
        \fillwithdottedlines{17mm}
    \end{parts}

\question
    It is known that $\tan \alpha = 1.4$, where $0^{\circ} < \alpha < 90^{\circ}$.
    \begin{parts}
        \part[2] Find the exact value of $\mathrm{cosec}^{2} \alpha$.
        \fillwithdottedlines{25mm}
        \part[2] Find the exact value of $ \cos \left(270^{\circ} - \alpha \right)$.
        \fillwithdottedlines{25mm}
    \end{parts}

\newpage

\question
    A stunt pilot flies at an average speed of 30 m/s. To break in the engine from point $A$, he flies for 50 seconds on a bearing of $134^{\circ}$ to point $B$, turns to fly on a bearing of $250^{\circ}$ for 2 minutes to point $C$, then turns to fly directly to point $A$ as shown in the diagram below.

    \begin{center}
        \begin{tikzpicture}[>=stealth]
            \draw[dashed,] (-3.084, 2.1580) -- (-1.084, 2.1580);
            \draw[dashed, <-] (-2.084, 3.1580)node[above]{N} -- (-2.084, 1.1580);
            \draw [] (0,0) -- (-2.084, 2.1580) node[above left]{$A$};
            \draw [] (-2.084, 2.1580)  -- (-5.1882, -4.118)node[below left]{$C$};
            \draw (-5.1882, -4.118) -- (0,0) node[above right]{$B$};
        \end{tikzpicture}
    \end{center}

    \begin{parts}

    \part[2] Find the distance between points $A$ and $C$, correct to one decimal place.
    \fillwithdottedlines{49mm}

    \part[3]
    Find  the bearing of $C$ from $A$, correct to the nearest degree.
    \fillwithdottedlines{49mm}
    
    \end{parts}

    \newpage

\question The diagram below shows the region, $ABCD$, covered by the $35$\,cm blade on a the end of a $50$\,cm windscreen wiper that makes an angle of $\dfrac{6 \pi}{11}$.
    \begin{center}
        \begin{tikzpicture}[scale=1.3]
            
            \fill[gray!35] (0,0) -- (2.64512, 2.29201) -- (2.64512, 2.29201) arc [start angle=40.909, delta angle=98.182, radius=3.5cm] -- (-2.64512, 2.29201) -- (0,0) -- cycle;

            \fill[white] (0,0) -- (1.1336, 0.9823 ) -- (1.1336, 0.9823 ) arc [start angle=40.909, delta angle=98.182, radius=1.5cm] -- (-1.1336, 0.9823 ) -- (0,0) -- cycle;

            \draw[] (0,0) -- node[below right]{$15$\,cm}(1.1336, 0.9823 )
            node[below right]{$C$};
            \draw[] (0,0) -- (-1.1336, 0.9823 )
            node[below left]{$D$};

            \draw [] (1.1336, 0.9823 ) -- (2.64512, 2.29201) node[right]{$B$};
            \draw[line width=1mm] (-1.1336, 0.9823 ) -- node[below left]{$35$\,cm}(-2.64512, 2.29201) node[left]{$A$};

            \draw (1.1336, 0.9823) arc [start angle=40.909, delta angle=98.182, radius=1.5cm];
            \draw (2.64512, 2.29201) arc [start angle=40.909, delta angle=98.182, radius=3.5cm];

            \coordinate[label=above:$\frac{6\pi}{11}$] (O) at (0,0.15);
            
        \end{tikzpicture}
    \end{center}

    \begin{parts}
    \part[1] Find the length of the arc $AB$, correct to two decimal places.
    \fillwithdottedlines{25mm}
    \part[2] Find the area of the region $ABCD$, correct to two decimal place.
    \fillwithdottedlines{49mm}
    \end{parts}

\question[3] Prove that $\sec \gamma +\tan \gamma + \cot \gamma = \dfrac{1 + \sin \gamma}{\sin \gamma \cos \gamma}$
\fillwithdottedlines{49mm}

\question[2] Simplify $\dfrac{\cos ^{2} \theta}{1 + \sin \theta} + \dfrac{\cos ^{2} \theta}{1 - \sin \theta}$.
\fillwithdottedlines{41mm}

\question Solve for $x$, where $0 \le x < 2\pi$:
    \begin{parts}
    \part[2] $\cos x = - \dfrac{1}{\sqrt{2}}$
    \fillwithdottedlines{41mm}
    \part[3] $2 \cos ^{2} x = \sin x +1$
    \fillwithdottedlines{49mm}
    \part[3] $\sin x - \sqrt{3} \cos x = 0$
    \fillwithdottedlines{49mm}
    \end{parts}


\newpage

\question[4]
    Two surveyors, $A$ and $B$, stand on level ground to observe a large tree. Surveyor $A$ sights the top of the tree on an angle of elevation of $52^{\circ}$. 
    Surveyor $B$ sights the top of the tree on an angle of elevation of $46^{\circ}$.
    Surveyor $B$ is 27 metres away from Surveyor $A$ on a bearing of $47^{\circ}$. The tree is directly north of surveyor $A$.
    Find the height, $h$, of the tree correct to the nearest metre.
    \makeemptybox{75mm}
    \fillwithdottedlines{96mm}

\end{questions}
\nomorequestions

\vspace*{35mm}
\begin{center}
    \textbf{\textit{The End}}
\end{center}

    
\end{document}
